%-----------------------------------------------------------------------------%
% Informasi Mengenai Dokumen
%-----------------------------------------------------------------------------%
% 
% Judul laporan. 
\var{\judul}{Pembanguna Korpus Pasangan Kata Relasi Hipernim-Hiponim Menggunakan \textit{Pattern}}
% 
% Tulis kembali judul laporan, kali ini akan diubah menjadi huruf kapital
\Var{\Judul}{PEMBANGUNAN KORPUS PASANGAN KATA RELASI HIPERNIM-HIPONIM MENGGUNAKAN \textit{PATTERN}}
% 
% Tulis kembali judul laporan namun dengan bahasa Ingris
\var{\judulInggris}{Pembanguna Korpus Pasangan Kata Relasi Hipernim-Hiponim Menggunakan \textit{Pattern}}

% 
% Tipe laporan, dapat berisi Skripsi, Tugas Akhir, Thesis, atau Disertasi
\var{\type}{Skripsi}
% 
% Tulis kembali tipe laporan, kali ini akan diubah menjadi huruf kapital
\Var{\Type}{Skripsi}
% 
% Tulis nama penulis 
\var{\penulis}{Made Nindyatama Nityasya}
% 
% Tulis kembali nama penulis, kali ini akan diubah menjadi huruf kapital
\Var{\Penulis}{Made Nindyatama Nityasya}
% 
% Tulis NPM penulis
\var{\npm}{1306381622}
% 
% Tuliskan Fakultas dimana penulis berada
\Var{\Fakultas}{Ilmu Komputer}
\var{\fakultas}{Ilmu Komputer}
% 
% Tuliskan Program Studi yang diambil penulis
\Var{\Program}{Ilmu Komputer}
\var{\program}{Ilmu Komputer}
\var{\programEng}{Computer Science}
% 
% Tuliskan tahun publikasi laporan
\Var{\bulanTahun}{Juli 2017}
% 
% Tuliskan gelar yang akan diperoleh dengan menyerahkan laporan ini
\var{\gelar}{Sarjana Ilmu Komputer}
% 
% Tuliskan tanggal pengesahan laporan, waktu dimana laporan diserahkan ke 
% penguji/sekretariat
\var{\tanggalPengesahan}{xx Juni 2013} 
% 
% Tuliskan tanggal keputusan sidang dikeluarkan dan penulis dinyatakan 
% lulus/tidak lulus
\var{\tanggalLulus}{x Juli 2013}
% 
% Tuliskan pembimbing 
\var{\pembimbing}{Mirna Adriani, Dra, Ph.D.}
\var{\pembimbingdua}{Rahmad Mahendra, S.Kom., M.Sc.}
% 
% Alias untuk memudahkan alur penulisan paa saat menulis laporan
\var{\saya}{Made}

%-----------------------------------------------------------------------------%
% Judul Setiap Bab
%-----------------------------------------------------------------------------%
% 
% Berikut ada judul-judul setiap bab. 
% Silahkan diubah sesuai dengan kebutuhan. 
% 
\Var{\kataPengantar}{Kata Pengantar}
\Var{\babSatu}{Pendahuluan}
\Var{\babDua}{Tinjauan Pustaka}
\Var{\babTiga}{Rancangan Metodologi}
\Var{\babEmpat}{Implementasi}
\Var{\babLima}{Evaluasi dan Hasil}
\Var{\babEnam}{Kesimpulan dan Saran}

\var{\tagHypernym}{\fontfamily{qcr}\selectfont<hypernym>}
\var{\tagHyponym}{\fontfamily{qcr}\selectfont<hyponym>}
\var{\tagStart}{\fontfamily{qcr}\selectfont<start>}
\var{\tagEnd}{\fontfamily{qcr}\selectfont<end>}
