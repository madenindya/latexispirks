%!TEX root = skripsi.tex
%-----------------------------------------------------------------------------%
\chapter{\babDua}
%-----------------------------------------------------------------------------%
Pada bab ini, dijelaskan mengenai studi literatur yang dilakukan. Studi literatur yang dilakukan digunakan sebagai dasar konsep dan teknik penelitian. Dipaparkan pula berbagai istilah dan metode yang digunakan dalam penelitian.

%-----------------------------------------------------------------------------%
\section{WordNet}
%-----------------------------------------------------------------------------%
WordNet adalah kamus leksikal yang tersimpan secara digital dan digunakan untuk berbagai keperluan komputasi (Miller, 1995). Pembuatan WordNet dilatarbelakangi keperluan mendapatkan \textit{sense} atau arti semantik suatu kata. Informasi tersebut perlu disimpan dan dapat dibaca oleh mesin. WordNet pertama dibuat oleh George A. Miller pada tahun 1995 berbasis Bahasa Inggris dan sekarang dikenal dengan nama Princeton WordNet (PWN). WordNet menyimpan informasi dalam bentuk database dimana setiap entry-nya adalah pasangan \textit{synset} dan arti semantiknya (\textit{sense}). Set sinonim (\textit{synset}) adalah himpunan kata yang memiliki arti yang sama atau saling berelasi \textit{synonym}. 

WordNet mengandung beberapa kelas kata seperti kata benda (\textit{noun}), kata kerja (textit{verb}), kata sifat (textit{adjective}), dan kata keterangan (\textit{adverb}). WordNet juga menyimpan informasi mengenai relasi semantik antar \textit{synset}. Relasi yang disimpan adalah \textit{synonymy}, \textit{antonymy}, \textit{hyponymy}, \textit{hypernym}, \textit{meronymy}, \textit{holonymy}, \textit{troponymy}, dan \textit{entailment}.

%-----------------------------------------------------------------------------%
\subsection{Indonesian WordNet}
%-----------------------------------------------------------------------------%

%-----------------------------------------------------------------------------%
\subsection{Klasifikasi X}
%-----------------------------------------------------------------------------%
Figure dalam enum dan dua sitasi sekaligus \citep{book.buyya,book.sterling-jones} :  
\begin{enumerate}
\item \bi{Bold Italic} \\
Penjelasan....... Untuk gambarannya dapat dilihat di Gambar \ref{fig:neumann}.

\begin{figure}
	\centering
	\includegraphics[height=0.65\textwidth,width=0.6\textwidth]
		{pics/neumann.pdf}
	\caption{Arsitektur klasik \f{von Neumann}}
	\label{fig:neumann}
\end{figure}
\vspace{-1.2cm}
\begin{center}
{\small Sumber gambar terinspirasi dari: \citep{buku.pressman}}
\end{center} 

\item \bi{Sesuatu banget} \\
Penjelasan.......
\end{enumerate}
\paragraph{}
%-----------------------------------------------------------------------------%
\section{\f{Section in Eng}}
%-----------------------------------------------------------------------------%
Hal pertama yang mungkin ditanyakan adalah bagaimana membuat huruf tercetak tebal, miring, atau memiliki garis bawah. 
Pada Texmaker, Anda bisa melakukan hal ini seperti halnya saat mengubah dokumen dengan LO Writer. 
Namun jika tetap masih tertarik dengan cara lain, ini dia: 

\begin{itemize}
	\item \bo{Bold} \\
		Gunakan perintah \bslash textbf$\lbrace\rbrace$ atau 
		\bslash bo$\lbrace\rbrace$. 
	\item \f{Italic} \\
		Gunakan perintah \bslash textit$\lbrace\rbrace$ atau 
		\bslash f$\lbrace\rbrace$. 
	\item \underline{Underline} \\
		Gunakan perintah \bslash underline$\lbrace\rbrace$.
	\item $\overline{Overline}$ \\
		Gunakan perintah \bslash overline. 
	\item $^{superscript}$ \\
		Gunakan perintah \bslash $\lbrace\rbrace$. 
	\item $_{subscript}$ \\
		Gunakan perintah \bslash \_$\lbrace\rbrace$. 
\end{itemize}

Perintah \bslash f dan \bslash bo hanya dapat digunakan jika package 
uithesis digunakan. 
%-----------------------------------------------------------------------------%
\subsection{Pengertian \f{Section in Eng}}
%-----------------------------------------------------------------------------%

%-----------------------------------------------------------------------------%
\subsection{Next Subsection \f{Section in Eng}}
%-----------------------------------------------------------------------------%

%-----------------------------------------------------------------------------%
\section{\f{Keatas lagi}}
%-----------------------------------------------------------------------------%
Contoh cite yang ga ada \cite{gaib}. Cite author \citeauthor{article.rebecca},cite tahun \citeyear{article.treese}, cite mention \cite{adin.experiment}, dan cite di akhir kalimat \citep{techreport.nist}.
%-----------------------------------------------------------------------------%
\subsection{\f{Masuk lagi}}
%-----------------------------------------------------------------------------%
Footnote example nih : MPICH \footnote{\url{http://www.mpich.org/}}, LAM/MPI \footnote{\url{http://www.lam-mpi.org/}}, dan OpenMPI \footnote{\url{www.open-mpi.org}} \citep{article.mcguire}. MPI-3 sedang dalam tahap perencanaan \footnote{\url{http://meetings.mpi-forum.org/MPI_3.0_main_page.php}}. Fungsi-fungsi tersebut berada di tabel \ref{tab:mpifund}. (Contoh tabel).

\begin{table}
	\centering
	\caption{Fungsi fundamental MPI}
	\label{tab:mpifund}
	\begin{tabular}{|c|c|c|}
	\hline
	\rowcolor{headertbl}	
	\hline No. & Nama Fungsi & Penjelasan \\ 
	\hline 1 & MPI\_Init & Memulai kode MPI \\ 
	\hline 2 & MPI\_Finalize & Mengakhiri kode MPI \\ 
	\hline 3 & MPI\_Comm\_size & Menentukan jumlah proses \\ 
	\hline 4 & MPI\_Comm\_rank & Menentukan label proses \\ 
	\hline 5 & MPI\_Send & Mengirim pesan \\ 
	\hline 6 & MPI\_Recv & Menerima pesan \\ 
	\hline
	\end{tabular}
\end{table}
\begin{center}
{\small Sumber tabel: taro sitasi disini, if i were u}
\end{center}
