%-----------------------------------------------------------------------------%
\chapter{\babEnam}
%-----------------------------------------------------------------------------%

%---------------------------------------------------------------
\section{Kesimpulan}
%---------------------------------------------------------------
Korpus pasangan kata relasi semantik dibutuhkan untuk digunakan dalam berbagai penelitian. Relasi kata hipernim-hiponim adalah relasi yang menghubungkan antara kata yang lebih umum dan kata yang lebih khusus. Proses pembangunan korpus dapat dilakukan dengan berbagai metode, salah satu yang cukup sering dan efisien adalah dengan pendekatan berbasis \textit{pattern} yaitu \textit{pattern matching} dan \textit{pattern extraction}. Metode berbasis \textit{pattern} dengan memanfaatkan korpus Wikipedia telah banyak dilakukan dalam beberapa penelitian sebelumnya. 

Pembangunan korpus dilakukan secara bertahap menggunakan pendekatan \textit{semi-supervised learning}. Secara umum, terdapat empat tahap utama dalam aristektur \textit{semi-supervised} yang diusulkan. Pertama adalah \textit{pre-processing} data yang terdiri dari pembentukan \textit{seed} dan pengolahan data Wikipedia. Kemudian \textit{pattern} dibentuk dengan melakukan \textit{sentence tagging} dari \textit{seed} ke kalimat Wikipedia dan dilanjutkan dengan \textit{pattern extraction} dari kalimat yang sudah memiliki \textit{tag}. Selanjutnya \textit{pair} baru diekstrak menggunakan \textit{pattern} yang sudah terbentuk terhadap korpus kalimat Wikipedia. Proses evaluasi dilakukan secara kolektif pada akhir eksperimen. Dalam implementasi \textit{cycle semi-supervised}, banyak parameter yang didefinisikan sendiri berdasarkan pengamatan kualitatif.

Setelah melalui proses pengumpulan \textit{seed}, pembentukkan \textit{pattern}, dan filterisasi, diambil lima \textit{pattern} terbaik dimana dibentuk oleh 140 \textit{seed} unik. Walaupun \textit{seed} yang diambil  sedikit, \textit{seed} tersebut dapat dikatakan baik dimana akurasinya mencapai 0.9. Hal ini menunjukan bahwa masih terdapat beberapa \textit{seed} salah yang dihasilkan akibat kekurangan yang dimiliki oleh WordNet Bahasa.

\textit{Pattern} yang dihasilkan oleh sistem masih sangat banyak yang tidak semua merepresentasikan relasi hipernim-hiponim yang umum. Jika dibandingkan dengan \textit{pattern} manual, \textit{pattern} yang tergolong \textit{exact match} sangat sedikit. Walau masih memiliki kekurangan, \textit{pattern} yang terpilih untuk digunakan pada proses \textit{pattern matching} dapat dikatakan baik karena berhasil menghasilkan korpus pasangan kata yang memiliki akurasi cukup tinggi. Selain itu, jika dilihat dari \textit{pattern} yang digunakan dalam proses ekstraksi \textit{pair}, \textit{precision} cukup tinggi dengan nilai terbesar mencapai 0.875. \textit{Pattern} adalah hal utama yang mempengaruhi kualitas korpus pasangan kata yang dihasilkan. \textit{Pattern} yang baik dapat menghasilkan korpus dengan nilai \textit{precision} tinggi.

Selanjutnya, \textit{pair} yang berhasil diekstraksi berjumlah banyak, sayangnya jumlahn yang diyakini benar dapat dikatakan sedikit. Untuk \textit{pair} yang lolos namun bernilai salah, kategori yang banyak dijumpai dalam \textit{pair} adalah pasangan kata merupakan relasi bukan hipernim-hiponim.Jumlah \textit{pair} yang terekstrak dapat dipengaruhi oleh proses pembobotan, metode filterisasi, \textit{stopping condition}, dan parameter arsitektur \textit{semi-supervised} lainnya. Sementara akurasi \textit{pair} lebih dipengaruhi oleh \textit{pattern} yang digunakan untuk ekstraksi. 

Penelitian ini memperlihatkan bahwa metode \textit{pattern analysis} yang terdiri dari proses \textit{pattern matching} dan \textit{extraction} dapat digunakan untuk mengekstraksi kata Bahasa Indonesia. Metode ini menghasilkan \textit{pair} yang cukup baik dimana untuk setiap eksperimen akurasinya melebihi 0.80. Penelitian pembangunan korpus pasangan kata relasi semantik hipernim-hiponim ini memiliki potensi untuk dikembangkan. Perlu dilakukan berbagai modifikasi parameter maupun eksperimen lain yang dapat meningkatkan kualitas \textit{pattern} dan \textit{pair} yang dihasilkan.

%---------------------------------------------------------------
\section{Saran}
%---------------------------------------------------------------
Dari hasil penelitian ini, berikut adalah beberapa saran yang dapat diberikan untuk penelitian berikutnya.

\begin{enumerate}
  \item Proses pembentukan kata yang merupakan \textit{multi token} saat ini masih sangat sederhana yaitu hanya dengan menggabungkan kata-kata yang memiliki \textit{POS tag} sama. Selanjutnya dapat diteliti cara yang lebih baik sehingga kata \textit{multi token} yang dihasilkan tidak mengandung token berlebih.
  \item Mengubah perhitungan pembobotan \textit{pair}. Proses pembobotan yang digunakan saat ini hanya memperhatikan jumlah \textit{pattern} yang membentuk \textit{pair} dan memanfaatkan pemodelan \textit{word embedding} untuk mengetahui kemiripan antar kata kemudian dicari rata-ratanya. Pembobotan ini didefinisikan sendiri berdasarkan analisis kualitatif yang dilakukan, namun belum diketahui apakah pembobotan ini sudah ideal. Penelitan selanjutnya dapat mengubah proses pembobotan yang lebih baik dalam merepresenstasikan kebaikan suatu \textit{pair}.
  \item Pembentukan \textit{pattern} yang lebih baik. \textit{Pattern} yang saat ini terbentuk hanya dapat menyimpan informasi mengenai tepat satu pasangan kata hipernim-hiponim. Sementara, jika dilihat dari \textit{pattern} yang dibentuk secara manual, banyak \textit{pattern} yang mengandung lebih dari satu \textit{tag} hiponim. Pembentukan \textit{pattern} perlu dimodifikasi untuk dapat menghasilkan \textit{pattern} tersebut. Dilihat dari proses pembentukan \textit{pattern}, cara-cara lain yang dapat dicoba adalah menggabungkan pendekatan yang memperhatikan $n$ kata diawal dan $n$ kata diakhir. Selanjutnya dari seluruh \textit{pattern} leksikal yang terbentuk dapat dilakukan generalisasi sehingga \textit{pattern} yang dihasilkan lebih representatif. Proses generalisasi \textit{pattern} dapat memanfaatkan informasi \textit{POS tag} kata-kata dalam \textit{pattern} tersebut.
  \item Modifikasi proses pembentukkan \textit{pattern} unik. Proses yang digunakan dalam penelitian ini adalah usulan yang dilihat berdasarkan pengamatan kualitatif. Cara lain untuk mengatasi masalah ambiguitas \textit{pattern} adalah dengan menghitung \textit{pattern} mana diantara dua \textit{pattern} ambigu yang lebih sering dibentuk.
  \item Ekstraksi pasangan kata relasi lain. Dari penelitian ini, dilihat bahwa tidak semua \textit{pattern} yang dihasilkan merepresentasikan relasi hipernim-hiponim dengan baik. Beberapa \textit{pattern} terlihat lebih cocok merepresentasikan relasi semantik lain. \textit{Pair} salah yang dihasilkanpun banyak yang merupakan \textit{pair} salah namun mengandung relasi semantik lain. Hal ini dapat disebabkan satu \textit{pattern} merepresentasikan beberapa relasi lain. Penelitian yang secara sekaligus mengekstrak pasangan kata untuk beberapa relasi semantik dapat dilakukan untuk membandingkan \textit{pattern} antar relasi tersebut.
  \item Memvalidasi fungsi pembobotan yang digunakan. Penelitian ini menggunakan fungsi pembobotan berdasarkan pengamatan kualitatif dan beberapa percobaan kombinasi terbaik. Fungsi yang dibentuk dapat dilakukan dengan menjalankan fungsi tersebut pada data pasangan kata relasi yang sudah diketahui benar seperti data \textit{seed}. Dengan begitu, fungsi yang digunakan dapat lebih terjamin benar.
  \item Mengembangkan arsitektur \textit{semi-supervised} yang lebih baik. Arsitektur \textit{semi-supervised} yang digunakan saat ini dikembangkan sendiri dengan metode \textit{Bootstrapping} sebagai dasarnya. Pengembangan selanjutnya dapat memodifikasi arsitektur tersebut dengan mengubah parameter-parameter yang didefinisikan sendiri seperti jumlah \textit{pattern} yang digunakan tiap iterasi, nilai \textit{threshold}, metode filterisasi, serta \textit{stopping condition}.
  \item Jika evaluasi dilakukan dengan cara \textit{sampling}, sebagiknya dilakukan tidak hanya sekali. Penelitian hanya melakukan \textit{sampling} sekali sehingga hasil akurasi \textit{pair} dipengaruhi data yang menjadi sampel.
  \item Melakukan penelitian untuk mengintegrasikan korpus pasangan kata relasi yang dihisilkan ke dalam WordNet Bahasa Indonesia agar dapat tersimpan dengan baik dan mudah digunakan untuk penelitian selanjutnya.
\end{enumerate}

