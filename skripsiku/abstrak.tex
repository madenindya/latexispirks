%
% Halaman Abstrak
%
% @author  Andreas Febrian
% @version 1.00
%

\chapter*{Abstrak}

\vspace*{0.2cm}

\noindent \begin{tabular}{l l p{10cm}}
	Nama&: & \penulis \\
	Program Studi&: & \program \\
	Judul&: & \judul \\
\end{tabular} \\ 

\vspace*{0.5cm}

\noindent 
\\ \textit{Word relation extraction} adalah salah satu penelitian di bidang NLP yang bertujuan untuk mengekstrak kata berdasarkan relasi yang didefinisikan. Korpus pasangan kata relasi dibutuhkan untuk menunjang berbagai penelitian selanjutnya. Korpus tersebut umumnya disimpan dalam kamus digital seperti WordNet Bahasa Inggris. Sayangnya untuk Bahasa Indonesia masih banyak kekurangannya. Penelitian ini dilakukan dengan tujuan membangun korpus pasangan kata relasi secara otomatis. Penelitian lain yang sama telah dilakukan dengan berbagai cara. Pendekatan yang dipilih untuk digunakan dalam penelitian ini adalah menggunakan \textit{pattern extraction} dan \textit{pattern matching}. Pembangunan korpus sendiri dilakukan secara bertahap dengan metode \textit{semi-supervised learning}. Relasi yang diteliti adalah relasi semantik hipernim-hiponim dengan menggunakan Wikipedia sebagai data sumber. Pada akhir penelitian, terbentuk korpus dengan total paling banyak 3493 pasangan kata. Akurasi untuk setiap eksperimen yang diujikan selalu lebih besar dari 0.8. Hal tersebut menunjukkan bahwa penggunaan metode \textit{pattern analysis} dengan data Wikipedia memiliki potensi untuk menghasilkan data berukuran besar dan berkualitas baik.

\vspace*{0.2cm}

\noindent Kata Kunci: \\ 
\noindent relasi kata, \textit{pattern extraction}, \textit{pattern matching}, \textit{semi-supervised}, hipernim-hiponum, korpus\\ 

\newpage
