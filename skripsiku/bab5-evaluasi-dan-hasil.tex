%-----------------------------------------------------------------------------%
\chapter{\babLima}
%-----------------------------------------------------------------------------%
Bab ini menjelaskan mengenai hasil untuk setiap tahapan penelitian, deskripsi percobaan yang telah dilakukan, serta evaluasi dan hasil terkait percobaan tersebut.

% 5.1
%-----------------------------------------------------------------------------%
\section{Hasil Pengolahan Data Wikipedia}
%-----------------------------------------------------------------------------%
Sebelum digunakan sebagai korpus masukan dalam proses \textit{pattern matching} dan \textit{extraction}, data Wikipedia perlu dikumpulkan dan diproses terlebih dahulu sehingga memenuhi format yang diinginkan.

%-----------------------------------------------------------------------------%
\subsection{Pengumpulan Data Wikipedia}
%-----------------------------------------------------------------------------%
Korpus teks dokumen utama dalam penelitian ini adalah artikel Wikipedia. Korpus diunduh dari situs Wikimedia. Berkas yang digunakan adalah idwiki-20170201-pages-article-multistream.xml.bz2, diunduh pada 20 Februari 2017. Berkas tersebut berukuran 398.6 MB dan merupakan data artikel Wikipedia Bahasa Indonesia hingga tanggal 1 Februari 2017. Setelah di-\textit{extract}, ukuran asli berkas XML tersebut adalah 1.9 GB. Berkas tersebut mengandung seluruh \textit{tag} identitas Wikipedia dan ditulis dalam format metadata Wikipedia. 

%-----------------------------------------------------------------------------%
\subsection{Ekstraksi Teks}
%-----------------------------------------------------------------------------%
Proses ekstraksi teks dilakukan karena tidak seluruh bagian teks digunakan sebagai data penelitian. Data wikipedia yang ingin digunakan hanya isi artikel. Korpus Wikipedia diekstrak menggunakan WikiExtractor untuk menghilangkan \textit{tag} yang tidak digunakan. Selain itu, artikel Wikipedia juga perlu dibersihkan dari simbol-simbol metadata. Setelah dilakukan proses ini, total ukuran berkasi Wikipedia menjadi 424 MB.

%-----------------------------------------------------------------------------%
\subsection{Pembentukan Kalimat}
%-----------------------------------------------------------------------------%
Data hasil ekstraksi memisahkan satu paragraf untuk setiap baris, sementara format yang diinginkan adalah satu baris merepresentasikan satu kalimat. Digunakan program \textit{sentence splitter} untuk memisahkan kalimat dalam berkas Wikipedia. Kemudian, dilanjutkan ke dalam proses \textit{rule-based formatter} sehingga memberi hasil kalimat yang sudah sesuai format yang ditentukan. Proses tersebut menghasilkan 3.696.339 kalimat dengan total ukuran berkas 431 MB. Berkas ini digunakan sebagai masukan proses \textit{pattern extraction}. 

%-----------------------------------------------------------------------------%
\subsection{Hasil POS Tagging Kalimat}
% TODO: tambahin hasil-hasil
%-----------------------------------------------------------------------------%
Kalimat-kalimat yang telah dibentuk, dimasukan ke dalam program POS Tagger, sehingga dihasilkan kalimat yang setiap tokennya memiliki \textit{tag} berdasarkan kelas kata. Total ukuran berkas adalah 623 MB. Berkas ini digunakan sebagai masukan proses \textit{pattern matching}. 

%-----------------------------------------------------------------------------%
\subsection{Pemodelan Word Embedding}
%-----------------------------------------------------------------------------%
Pembuatan model \textit{word embedding} menggunakan korpus Wikipedia yang sudah berbentuk kalimat. Proses ini menghasilkan tiga berkas model (tabel \ref{table:modelWE}). Berkas hasil pemodelan berukuran besar karena menggunakan seluruh kalimat yang ada dalam korpus Wikipedia.

\begin{table}
  \centering
    \caption{Berkas model \textit{word embedding}}
    \label{table:modelWE}
    \begin{tabular}{|c|c|c|}
      \hline
        Nama Berkas                       & Jenis Berkas & Ukuran Berkas \\ \hline
        model-word2vec                    & File         & 306.1 MB      \\ \hline
        model-word2vec.syn0.npy           & NPY File     & 1.8 GB        \\ \hline
        model-word2vec.syn1neg.npy        & NPY File     & 1.8 GB        \\ \hline
    \end{tabular}
\end{table}

% 5.2
%-----------------------------------------------------------------------------%
\section{Pengumpulan Seed}
\label{bab-pengumpulanseed}
%-----------------------------------------------------------------------------%
Pasangan kata relasi hipernim-hiponim awal diambil dari korpus yang dimiliki oleh WordNet Bahasa. Setelah melalui proses pembentukan \textit{seed}, berikut adalah jumlah \textit{seed} yang dihasilkan berdasarkan jenis filterisasi. Kedua tipe \textit{seed} tersebut digunakan proses pembentukan pattern pada iterasi pertama. 

\begin{table}
  \centering
    \caption{Jumlah \textit{seed} hasil ekstraksi}
      \label{table:jumlahSeed}
      \begin{tabular}{|c|c|c|}
        \hline
          No & Jenis Filterisasi & Jumlah Seed \\ \hline
          1. & lema sama         & 8.985       \\ \hline
          2. & \textit{strict}   & 8.602       \\ \hline
    \end{tabular}
\end{table}

Walau sudah melakukan proses filterisasi sebagai upaya mengurangi \textit{error} dan ambiguitas yang terjadi pada proses pemenentukan \textit{seed} serta untuk meningkatkan kulatias \textit{seed}, masih ada hambatan yang belum dapat diatasi dalam penelitian ini. Beberapa kelemahan dari \textit{seed} awal yang dihasilkan adalah sebagai berikut.

\begin{itemize}
  \item \textit{Seed} yang mengandung kata bukan Bahasa Indonesia. Korpus yang ingin dibuat berdomain Bahasa Indonesia, namun \textit{seed} yang dihasilkan mengandung Bahasa Melayu atau Bahasa Indonesia lama. Beberapa kata bukan Bahasa Indonesia yang dihasilkan adalah 'had', 'bonjol', dan 'cecok'.
  \item Kesalahan semantik \textit{synset} dan lema Bahasa Indonesia. Beberapa \textit{synset} nltk memiliki lema Bahasa Indonesia yang kurang sesuai jika dilihat secara semantik. Sebagai contoh \textit{Synset('scholar.n.01')} dengan lemma Bahasa Indonesia \textit{{'buku\_harian', 'pelajar'}}. Dalam Bahasa Indonesia, 'buku\_harian' memiliki makna yang berbeda dengan 'pelajar'.
  \item Kesalahan lema Bahasa Indonesia untuk suatu \textit{synset} menyebabkan dihasilkannya \textit{seed} yang jika dievaluasi kualitatif oleh manusia dirasa kurang tepat. Contoh \textit{seed} yang tidak baik adalah dari pemetaan \textit{('sejarawan', Synset('historian.n.01')) => (['buku\_harian', 'pelajar'], Synset('scholar.n.01'))} dihasilkan \textit{seed} \textit{(sejarawan,buku harian)} dan \textit{(sejarawan,pelajar)}. \textit{Seed} (sejarawan,buku harian) adalah salah.
\end{itemize}

Walau terlihat masih banyak ditemukan kasus salah seperti di atas, beberapa pasangan kata yang terbentuk dapat diniali cukup baik. Beberapa \textit{seed} baik yang berhasil dibentuk adalah '(lori,truk)', '(kecap,bumbu)', '(itik,unggas)', dan '(kelelawar,mamalia)'.

% 5.3
%-----------------------------------------------------------------------------%
\section{Pembentukan Pattern Pertama}
%-----------------------------------------------------------------------------%
Iterasi pertama untuk seluruh percobaan selalu menggunakan \textit{seed} diatas, sehingga hasil untuk proses pembentukan \textit{pattern} sama. Berikut adalah detil hasil untuk \textit{sentence tagging} dan \textit{pattern extraction} pada iterasi pertama.

%-----------------------------------------------------------------------------%
\subsection{Sentence Tagging dengan Seed}
% TODO: hitung kalimat unik-nya
%-----------------------------------------------------------------------------%
Dari kedua tipe \textit{seed} hasil filterisasi, masing-masing digunakan untuk membentuk korpus kalimat yang memiliki \textit{tag} relasi hipernim-hiponim. Berikut adalah hasil proses \textit{sentence tagging} menggunakan kedua tipe \textit{seed}.

\begin{table}
  \centering
  \caption{Hasil \textit{sentence tagging} dengan \textit{seed}}
  \label{table:sentencetagging1}
  \resizebox{\textwidth}{!}{
  \begin{tabular}{|c|c|c|c|c|}
  \hline
  No. & Jenis Seed                        & Jumlah kalimat & Jumlah kalimat unik & Ukuran berkas \\ \hline
  1.  & Seed dengan filterisasi lema sama & 69.574         & 29.977       & 14 MB         \\ \hline
  2.  & Seed dengan filterisasi strict    & 64.718         & 25.378       & 13 MB         \\ \hline
  \end{tabular}
  }
\end{table}

\noindent Data pada tabel \ref{table:sentencetagging1} memperlihatkan bahwa banyak ditemukan kasus dimana satu kalimat mengandung lebih dari satu pasangan kata relasi. Sementara, jika dibandingkan dengan keseluruhan kalimat dalam Wikipedia, hanya sekitar 0.01 yang mengandung \textit{seed}. Hal ini dapat berarti banyak kalimat yang kurang merepresentasikan relasi atau banyak \textit{seed} yang kurang tepat. Seperti yang sudah dijelaskan pada bab \ref{bab-pengumpulanseed}, masih banyak \textit{seed} kurang baik yang terbentuk.

%-----------------------------------------------------------------------------%
\subsection{Hasil Pattern Extraction}
%-----------------------------------------------------------------------------%
Kedua korpus kalimat yang masing-masing menghasilkan daftar \textit{pattern}. Kedua daftar \textit{pattern} digabung dan diurutkan sesuai bobot. Dari 106 \textit{pattern} yang dihasilkan pada iterasi pertama, tabel \ref{table:pattern1} memaparkan lima \textit{pattern} terbaik yang digunakan untuk proses \textit{pattern matching}. Kelima \textit{pattern} tersebut dibentuk dari total 104 \textit{seed} yang langsung masuk ke korpus pasangan kata relasi.

\begin{table}
  \centering
  \caption{Pattern terbaik iterasi pertama}
  \label{table:pattern1}
  \begin{tabular}{|c|}
    \hline
      start <hyponym> adalah <hypernym> \\
      <hyponym> merupakan <hypernym> \\ 
      <hyponym> adalah <hypernym> yang \\
      <hypernym> seperti <hyponym> dan \\
      <hypernym> termasuk <hyponym> \\ \hline
  \end{tabular}
\end{table}

Kelima \textit{pattern} tersebut dibangun dengan total 140 \textit{seed}. Dari 140 \textit{seed} yang merepresentasikan kelima \textit{pattern}, dicari nilai akurasinya. Diambil 20 \textit{pair} secara acak sehingga satu sampel merepresentasikan 7 data asli, kemudian \textit{pair} tersebut dianotasi secara manual oleh tiga anotator. Total sampel yang benar adalah 18 data, sehingga dapat dikatakan bahwa akurasi \textit{pair} pertama yang masuk ke dalam korpus pasangan kata relasi adalah 0.9. Jika dilihat dari keseluruhan data, masih terdapat 14 \textit{pair} yang salah namun masuk ke dalam pasangan kata relasi. Hal ini berarti walau sudah dilakukan proses filterisasi, korpus \textit{seed} yang dihasilkan masih mengandung \textit{error}. 

% 5.4
%-----------------------------------------------------------------------------%
\section{Hasil Eksperimen}
%-----------------------------------------------------------------------------%
Bagian ini akan memaparkan penjelasan parameter untuk setiap eksperimen yang telah dilakukan dan hasil dari eksperimen tersebut. Setiap eksperimen dilalukan dengan memodifikasi parameter dalam arsitektur \textit{semi-supervised} yaitu mengubah nilai \textit{threshold} dalam filterisasi \textit{pair} baru atau mengubah tipe \textit{pair} yang diterima antar \textit{single token} dan \textit{multi token}.

%-----------------------------------------------------------------------------%
\subsection{Eksperimen 1}
%-----------------------------------------------------------------------------%
Eksperimen pertama dilakukan dengan nilai \textit{threshold} untuk filterisasi \textit{pair} baru sebesar 0.6. Percobaan pertama menerima seluruh tipe kata, baik yang merupakan \textit{multi token} maupun \textit{single token}. Tabel \ref{table:eksp1-pattern} menampilkan total \textit{pattern} yang dihasilkan dan digunakan untuk setiap iterasi. Sementara tabel \ref{table:eksp1-pair} menampilkan rincian jumlah \textit{pair} yang dihasilkan dan diterima untuk setiap iterasi.

\begin{table}
  \centering
  \caption{Pattern Hasil Eksperimen 1}
  \label{table:eksp1-pattern}
  \begin{tabular}{|c|c|c|}
  \hline
  Iterasi ke- & Total pattern & Pattern untuk ekstraksi                 \\ \hline
  1           & 106           & Pattern utama                           \\ \hline
  2           & 347           & <start> <hyponym> merupakan <hypernym>  \\ \hline
  3           & 395           & <hyponym> merupakan <hypernym> yang     \\ \hline
  4           & 425           & <hypernym> atau <hyponym>               \\ \hline
  \end{tabular} 
\end{table}

\begin{table}
  \centering
  \caption{Pair Hasil Eksperimen 1}
  \label{table:eksp1-pair}
  \begin{tabular}{|c|c|c|c|}
  \hline
  Iterasi ke-  & Total Pair & Pair baru & Korpus Pasangan kata relasi \\ \hline
  1            & 82409      & 799       & 939  \\ \hline
  2            & 104389     & 316       & 1255 \\ \hline
  3            & 108397     & 175       & 1430 \\ \hline
  4            & 108542     & 2         & 1432 \\ \hline
  \end{tabular} 
\end{table}

Data di atas memperlihatkan bahwa jumlah \textit{pair} yang diyakini benar atau nilai bobotnya melebihi \textit{threshold} dapat dikatakan sedikit jika dibandingkan dengan total \textit{pair} yang terekstraksi. Perbandingan total \textit{pair} yang masuk ke dalam korpus pasangan kata relasi dengan total \textit{pair} yang dihasilkan oleh sistem hanya 0.013. Untuk meningkatkan jumlah \textit{pair} yang masuk ke dalam korpus pasangan kata relasi, dilakukanlah eksperimen dimana nilai \textit{threshold} diturunkan.

% biar nyambung ke eksperimen 3
Selanjutnya, dari seluruh data sampel eksperimen pertama (tabel \ref{table:akurasi-all}), \textit{pair} yang kata hipernim dan hiponimnya merupakan \textit{single token} ada sejumlah 73 \textit{pair}. Setelah dianotasi, 63 diantaranya adalah \textit{pair} yang berlabel benar. Hal ini menunjukan akurasinya untuk \textit{pair} yang bertipe \textit{single token} adalah 0.86. Kemudian dari data tersebut, 44 \textit{pair} adalah kata hipernim-hiponim yang keduanya merupakan konsep. Hal tersebut memperlihatkan bahwa \textit{pair} yang kedua katanya adalah \textit{single token} memiliki kualitas yang jauh lebih baik dibanding \textit{multi token}. Untuk membuktikan hal tersebut, dilakukan eksperimen 3 yang hanya memperhatikan data yang merupakan \textit{single token}.

% \noindent \textit{Pair} yang dihasilkan dicari nilai akurasinya dengan mengambil sampel secara \textit{random}. Sampel tersebut kemudian dianotasi secara manual oleh anotator. Tabel memaparkan nilai akurasi \textit{pair} untuk setiap iterasi.
% \begin{table}
%   \centering
%   \caption{Akurasi Eksperimen 1}
%   \label{table:eksp1-akurasi}
%   \begin{tabular}{|c|c|c|c|c|}
%   \hline
%   Iterasi ke- & Sampel baru & Sampel benar & Akurasi sampel baru & Akurasi total \\ \hline
%   1           & 168         & 133          & 0.79                & 0.8 \\ \hline
%   2           & 59          & 45           & 0.76                & 0.79 \\ \hline
%   3           & 40          & 36           & 0.9                 & 0.81 \\ \hline
%   4           & 2           & 1            & 0.5                 & 0.81 \\ \hline
%   \end{tabular} 
% \end{table}
% Secara total, korpus pasangan kata relasi untuk eksperimen pertama menghasilkan \textit{pair} dengan akurasi kebenaran 0.81. Walau nilai akurasi cukup tinggi, namun jika dilihat berdasarkan jumlah data masih terdapat sekitar 277 \textit{pair} salah yang masuk ke dalam korpus pasangan kata relasi. Kesalahan \textit{pair} disebabkan oleh berbagai hal. Beberapa \textit{pair} tidak memiliki hubungan sama sekali dimana hal tersebut bisa terjadi karena ada kalimat yang memuat pattern leksikal namun kata tidak merepresentasikan relasi hipernim-hiponim. Kesalahan lain ketika \textit{pair} yang terbentuk saling berelasi namun bukan relasi hipernim-hiponim atau posisi kata yang tertukar.

%-----------------------------------------------------------------------------%
\subsection{Eksperimen 2}
%-----------------------------------------------------------------------------%
Melihat sedikitnya jumlah pattern yang dihasilkan dari eksperimen pertama, maka diputuskan untuk menurunkan sedikit nilai \textit{threshold}. Nilai \textit{threshold} untuk filterisasi \textit{pair} baru menjadi sebesar 0.55. Penurunan nilai yang hanya 0.05 dilakukan untuk menjaga kualitas \textit{pair} yang dihasilkan. Tabel \ref{table:eksp2-pattern} dan \ref{table:eksp2-pair} memaparkan hasil dari eksperimen kedua.

\begin{table}
  \centering
  \caption{Pattern Hasil Eksperimen 2}
  \label{table:eksp2-pattern}
  \begin{tabular}{|c|c|c|}
  \hline
    Iterasi ke- & Total pattern & Pattern baru terpilih \\ \hline
    1 & 106 & Pattern utama \\ \hline
    2 & 699 & <start> <hyponym> merupakan <hypernym> \\ \hline
    3 & 791 & <hyponym> merupakan <hypernym> yang \\ \hline
    4 & 842 & <hyponym> menjadi <hypernym> \\ \hline
  \end{tabular} 
\end{table}

\begin{table}
  \centering
  \caption{Pair Hasil Eksperimen 2}
  \label{table:eksp2-pair}
  \begin{tabular}{|c|c|c|c|}
  \hline
  Iterasi ke-  & Total Pair & Pair baru & Korpus Pasangan kata relasi \\ \hline
  1 & 82409 & 2053 & 2193 \\ \hline
  2 & 104389 & 878 & 3071 \\ \hline
  3 & 108397 & 413 & 3484 \\ \hline
  4 & 108737 & 9 & 3493 \\ \hline
  \end{tabular} 
\end{table}

\noindent Jika dibandingkan dengan eksperiment pertama, jumlah \textit{pair} yang nilainya melebihi \textit{threshold} bertambah hingga dua kali dari sebelumnya walaupun selisih nilai \textit{threshold} hanya 0.05. Ini berarti banyak \textit{pair} yang nilai bobotnya dekat dengan nilai batas tersebut. Selanjutnya, jika dilihat dari total \textit{pair} yang diekstrak, hingga iterasi ke-3 jumlahnya total \textit{pair} yang terekstrak sama. Perubahan terjadi pada iterasi ke-4 ketika \textit{pattern} yang digunakan berubah. Hal ini berarti jumlah total \textit{pair} yang terekstrak sepenuhnya bergantung dengan \textit{pattern} yang digunakan.
%
% \noindent Dari \textit{pair} yang dihasilkan, tabel \ref{table:eksp2-akurasi} menunjukan nilai akurasinya.
% \begin{table}
%   \centering
%   \caption{Akurasi Eksperimen 2}
%   \label{table:eksp2-akurasi}
%   \begin{tabular}{|c|c|c|c|c|}
%   \hline
%   Iterasi ke- & Sampel baru & Sampel benar & Akurasi sampel baru & Akurasi total \\ \hline
%   1 & 308 & 257 & 0.83 & 0.84 \\ \hline
%   2 & 129 & 106 & 0.82 & 0.83 \\ \hline
%   3 & 45 & 37 & 0.82 & 0.83 \\ \hline
%   4 & 9 & 9 & 1 & 0.84 \\ \hline
%   \end{tabular} 
% \end{table}

%-----------------------------------------------------------------------------%
\subsection{Eksperimen 3}
%-----------------------------------------------------------------------------%
Setelah menganalisis secara kualitatif \textit{pair} yang dihasilkan, banyak \textit{pair} yang kata hipernimnya adalah konsep sementara kata hiponimnya adalah \textit{instance}. Untuk relasi semantik hipernim-hiponim yang lebih umum, hubungan antar kata yang lebih umum adalah antar konsep. Untuk pasangan kata relasi hipernim-hiponim yang bertipe konsep-konsep, lebih banyak dijumpai jika kedua kata merupakan \textit{single token}. Selain it, pembentukan \textit{multi token} yang dilakukan belum sepenuhnya baik seperti banyaknya dijumpai kasus terbentuk kata yang kelebihan token. Eksperimen ini akan mencoba mengekstrak \textit{pair} yang kedua katanya adalah \textit{single token}. Nilai \textit{threshold} yang digunakan adalah 0.55.

\begin{table}
  \centering
  \caption{\textit{Pattern} Hasil Eksperimen 3}
  \label{table:eksp3-pattern}
  \begin{tabular}{|c|c|c|}
  \hline
    Iterasi ke- & Total pattern & Pattern baru terpilih \\ \hline
    1 & 106 & Pattern utama \\ \hline
    2 & 437 & <hyponym> adalah sebuah <hypernym> \\ \hline
  \end{tabular} 
\end{table}

\begin{table}
  \centering
  \caption{\textit{Pair} Hasil Eksperimen 3}
  \label{table:eksp3-pair}
  \begin{tabular}{|c|c|c|c|}
  \hline
  Iterasi ke-  & Total Pair & Pair baru & Korpus Pasangan kata relasi \\ \hline
  1 & 10267 & 298 & 438 \\ \hline
  2 & 11262 & 21 & 459 \\ \hline
  \end{tabular} 
\end{table}

\noindent Untuk \textit{pair} yang hanya \textit{single word} sudah pasti jumlahnya jauh lebih sedikit. Jika dibandingkan dengan dua eksperimen sebelumnya, total \textit{pattern} yang dihasilkan juga jauh lebih sedikit. Hal ini memperlihatkan bahwa jumlah pasangan kata yang digunakan mempengaruhi jumlah \textit{pattern} yang dihasilkan, dimana semakin banyak jumlah pasangan kata relasi maka semakin banyak pula \textit{pattern} yang dihasilkan.
%
% \noindent Hasil perhitungan akurasi untuk eksperimen ini dapat dilihat pada tabel \ref{table:eksp3-akurasi}.
% \begin{table}
%   \centering
%   \caption{Akurasi Eksperimen 3}
%   \label{table:eksp3-akurasi}
%   \begin{tabular}{|c|c|c|c|c|}
%   \hline
%   Iterasi ke- & Sampel baru & Sampel benar & Akurasi sampel baru & Akurasi total \\ \hline
%   1 & 78 & 67 & 0.86 & 0.87 \\ \hline
%   2 & 5 & 4 & 0.8 & 0.86 \\ \hline
%   \end{tabular} 
% \end{table}

%-----------------------------------------------------------------------------%
\subsection{Analisis Eksperimen}
% TODO: hapus yang berhubungan dengan akurasi
%-----------------------------------------------------------------------------%
Dari ketiga eksperimen yang telah dilakukan, berikut adalah beberapa hal yang dapat dilihat.
\begin{itemize}
  \item \textit{Pair} yang diyakini benar tergolong sedikit jika dibandingkan dengan total \textit{pair} yang terkestraksi.
  \item \textit{Pattern} yang sama akan menghasilkan total \textit{pair} yang terkestraksi sama. Pada eksperimen 1 dan 2, hingga iterasi 3, total \textit{pair} yang dihasilkan adalah sama akibat ektraksi menggunakan \textit{pattern} yang sama.
  \item Semakin banyak pasangan kata yang digunakan untuk membentuk \textit{pattern}, maka semakin banyak pula \textit{pattern} yang dihasilkan.
  \item Menurunkan sedikit nilai \textit{threshold} dapat menambah banyak \textit{pair} yang lolos filterisasi. Pada eksperimen 2, menurunkan \textit{threshold} sebesar 0.05 meningkatkan jumlah \textit{pair} hinga dua kali lipat. Hal tersebut juga menunjukan bahwa banyak \textit{pair} yang nilai \textit{threshold}-nya berada di ambang batas. 
  \item \textit{Stopping condition} dengan hanya menghitung jumlah \textit{pair} baru tidak menjamin bahwa sudah tidak ada \textit{pair} benar.   
\end{itemize}

% 5.5
%-----------------------------------------------------------------------------%
\section{Hasil Evaluasi Pair}
%-----------------------------------------------------------------------------%
Proses evaluasi \textit{pair} dilakukan hanya terhadap \textit{pair} yang diterima ke dalam korpus pasangan kata. Anotasi \textit{pair} dilakukan oleh tiga anotator berbeda dengan daftar \textit{pair} yang sama. Dari data hasil anotasi, akan dilihat akurasi tiap eksperimen yang telah dilakukan, akurasi berdasarkan \textit{pattern} yang digunakan, tingkat persetujuan antar anotator, kemudian menganalisis data hasil.

Total sampel yang dianotasi adalah 514 \textit{pair}, dimana sampel tersebut diambil dari gabungan ketiga percobaan. Anotator memberi nilai anotasi terhadap dua dimensi yaitu apakah \textit{pair} benar memiliki relasi semantik hipernim-hiponim dan mengelompokan berdasarkan kategorinya. Sayangnya, nilai yang diberikan antar anotator tidak selalu sama, untuk itu ingin diketahui tingkat persetujuan antar anotator. Tingkat persetujuan ketiga anotator dihitung menggunakan Fleiss' Kappa karena anotasi dilakukan oleh lebih dari dua anotator.
%Akibat keterbatasan sumber daya, tidak semua \textit{pair} yang dihasilkan dapat dianotasi secara manual. Untuk itu, dilakukan \textit{sampling} dari seluruh \textit{pair} unik yang dihasilkan. 
% Evaluasi dilakukan dengan melihat akurasi \textit{pair} berdasarkan hasil anotasi. 

%-----------------------------------------------------------------------------%
\subsection{Hasil Anotasi Pair}
%-----------------------------------------------------------------------------%
Data yang telah dianotasi digunakan untuk mengetahui akurasi korpus pasangan kata relasi yang dihasilkan. Tabel \ref{table:jumlah-bs} memperlihatkan perbandingan jumlah data yang dianotasi antar anotator. Kolum dalam tabel menyatakan jumlah data yang diberi label benar oleh n anotator. Misal `3 Benar' berarti kolum tersebut memperlihatkan jumlah data yang dianotasi benar oleh ketiga anotator. Kolum `1 Benar' memperlihatkan jumlah data yang dianotasi benar oleh hanya seorang anotator atau dapat diartikan juga jumlah data yang dilabeli benar oleh satu anotator dan salah oleh dua anotator.

\begin{table}
  \centering
  \caption{Perbandingan hasil anotasi \textit{pair} antar anotator}
  \label{table:jumlah-bs}
  \begin{tabular}{|c|c|c|c|}
  \hline
  3 Benar & 2 Benar & 1 Benar & 0 Benar \\ \hline
  345 & 83 & 52 & 33 \\ \hline
  \end{tabular} 
\end{table}

Dari data di atas, dapat dihitung akurasi kebenaran sampel. Data yang memiliki labelnya tidak sama untuk ketiga anotator seperti kolum `2 Benar' dan `1 Benar', nilai kebenaran diambil menggunakan sistem \textit{voting}. Akurasi total yang diadapatkan dari seluruh data sampel adalah 0.83.

Selanjutnya, untuk mengevaluasi parameter \textit{semi-supervised} yang dimodifikasi, perlu dicari dihitung akurasi untuk setiap eksperimen sehingga dapat dibandingkan. Tabel \ref{table:akurasi-all} memperlihatkan akurasi \textit{pair} untuk setiap eksperimen.

\begin{table}
  \centering
  \caption{Akurasi \textit{pair} untuk setiap eksperimen}
  \label{table:akurasi-all}
  \begin{tabular}{|c|c|c|c|c|}
  \hline
  Eksperimen & Total \textit{Pair} & Jumlah Sampel & Jumlah Sampel Benar & Akurasi \\ \hline
  1 & 1432 & 289 & 233 & 0.80 \\ \hline
  2 & 3493 & 511 & 427 & 0.84 \\ \hline
  3 & 459  & 103 & 89  & 0.86 \\ \hline
  \end{tabular} 
\end{table}

Data di atas menunjukan bahwa menurunkan nilai \textit{threshold} belum tentu memperburuk akurasi \textit{pair} yang dihasilkan. Hal ini berarti persebaran \textit{pair} yang salah belum secara jelas dapat diketahui dengan pembobotan yang saat ini digunakan. Selain itu dapat diketahui pula bahwa \textit{pair} yang hanya memperhatikan \textit{single token} memiliki kualitasi lebih baik dibanding \textit{pair} yang memperbolehkan \textit{multi token}.

%-----------------------------------------------------------------------------%
\subsection{Akurasi Pair berdasarkan Pattern}
%-----------------------------------------------------------------------------%
Evaluasi \textit{pair} dapat dilihat berdasarkan \textit{pattern} yang membentuk \textit{pair} tersebut. Tabel \ref{table:akurasi-pair-patt} memperlihatkan total \textit{pair} yang dihasilkan dan akurasi untuk setiap \textit{pattern}. \textit{Pattern} yang dianalisis dalam tabel tersebut hanya \textit{pattern} yang digunakan untuk proses ekstraksi \textit{pair}. Total \textit{pair} menyatakan jumlah \textit{pair} yang terbentuk menggunakan \textit{pattern} dalam seluruh eksperimen yang telah dilakukan. \textit{Pair} diterima menyatakan total \textit{pair} yang masuk ke dalam korpus pasangan kata relasi untuk seluruh eksperimen.

\begin{table}
  \centering
  \caption{Akurasi \textit{pair} berdasarkan \textit{pattern}}
  \label{table:akurasi-pair-patt}
  \resizebox{\textwidth}{!}{\begin{tabular}{|c|c|c|c|c|c|} 
  \hline
  \textit{Pattern} & Total \textit{pair} & \textit{Pair} diterima & Sampel & \textit{Pair} benar & Akurasi \\ \hline
  <hypernym> atau <hyponym> & 164 & 8 & 5 & 2 & 0.4 \\ \hline
  <hypernym> seperti <hyponym> dan & 136 & 38 & 7 & 7 & 1 \\ \hline
  <hypernym> termasuk <hyponym> & 238 & 69 & 19 & 19 & 1 \\ \hline
  <hyponym> adalah <hypernym> yang & 81892 & 2078 & 308 & 258 & 0.84 \\ \hline
  <hyponym> adalah sebuah <hypernym> & 1910 & 36 & 7 & 5 & 0.71 \\ \hline
  <hyponym> menjadi <hypernym> & 574 & 72 & 25 & 25 & 1 \\ \hline
  <hyponym> merupakan <hypernym> & 27095 & 1508 & 237 & 202 & 0.85 \\ \hline
  <hyponym> merupakan <hypernym> yang & 9715 & 754 & 132 & 114 & 0.86 \\ \hline
  <start> <hyponym> adalah <hypernym> & 83319 & 2160 & 316 & 264 & 0.84 \\ \hline
  <start> <hyponym> merupakan <hypernym> & 22618 & 1363 & 207 & 177 & 0.86 \\ \hline
  \end{tabular}}
\end{table}

Data di atas menunjukan sebagian besar \textit{pattern} yang terpilih untuk proses \textit{pattern matching} adalah \textit{pattern} yang baik karena menghasilkan kumpulan \textit{pair} dengan akurasi yang cukup tinggi. Namun, ada juga \textit{pattern} terpilih yang nilai akurasinya buruk yaitu \textit{pattern} `<hypernym> atau <hyponym>'.

%-----------------------------------------------------------------------------%
\subsection{Tingkat Persetujuan Anotator Pair}
%-----------------------------------------------------------------------------%
Dalam proses anotasi, tidak mungkin semua data dapat dianotasi ke dalam kategori yang sama oleh seluruh anotator. Perhitungan tingkat persetujuan dapat dilakukan dengan mencari nilai Fleiss' Kappa, Tingkat persetujuan diukur berdasarkan nilai anotasi data berlabel benar atau salah untuk seluruh sampel. Tiga anotator ($n=3$) memberi penilaian terhadap 514 data sampel ($N=514$) ke dalam dua label ($k=2$) yaitu benar atau salah. Tabel \ref{table:pair-kappa} memaparkan langkah-langkah hingga didapatkan nilai kappa. Sementara, tabel \ref{table:pair-kappa2} adalah perhitungan nilai kappa berdasarkan lima kategori ($k=5$) yang didefinisikan.

\begin{table}
  \centering
  \caption{Perhitungan tingkat persetujuan anotator berdasarkan label benar/salah \textit{pair}}
  \label{table:pair-kappa}
  \begin{tabular}{|c|c|}
  \hline
  $sum\,\,P_i$ & 423.33 \\ \hline
  $P_o$ & 0.82 \\ \hline
  $p_{True}$ & 0.81 \\ \hline
  $p_{False}$ & 0.19 \\ \hline
  $P_e$ & 0.70 \\ \hline
  $Kappa$ & 0.42 \\ \hline
  \end{tabular} 
\end{table}

\noindent Setelah diperoleh nilai $P_o$ dan $P_e$, nilai Kappa dapat dihitung menggunakan rumus 2.1. Tingkat persetujuan ketiga anotator berdasarkan nilai kebenaran \textit{pair} adalah 0.42. Nilai tersebut menunjukan bahwa tingkat persetujuan antar anotator tergolong \textit{moderate aggreement}.

\begin{table}
  \centering
  \caption{Perhitungan tingkat persetujuan anotator berdasarkan kategori \textit{pair}}
  \label{table:pair-kappa2}
  \begin{tabular}{|c|c|}
  \hline
  $sum\,\,P_i$ & 329.33 \\ \hline
  $P_o$ & 0.64 \\ \hline
  $p_{CC}$ & 0.29 \\ \hline
  $p_{IC}$ & 0.52 \\ \hline
  $p_{FF}$ & 0.07 \\ \hline
  $p_{FR}$ & 0.10 \\ \hline
  $p_{FP}$ & 0.02 \\ \hline
  $P_e$ & 0.37 \\ \hline
  \end{tabular} 
\end{table}

\noindent Nilai kappa yang dihasilkan jika dilihat dari anotasi berdasarkan kategori adalah 0.43. Hal tersebut menunjukan bahwa tingkat persetujuan untuk kasus ini sama seperti sebelumnya yaitu tergolong \textit{moderate agreement}.

%-----------------------------------------------------------------------------%
\subsection{Analisis Pair yang Dihasilkan}
%-----------------------------------------------------------------------------%
Dari hasil anotasi terebut, beberapa hal yang diketahui adalah sebagai berikut.
\begin{itemize}
  \item Kenaikan atau penurunan akurasi setiap iterasi hanya tergantung dari data yang digunakan sebagai sampel. 
  \item \textit{Pair} yang hanya terdiri dari \textit{single token} memiliki nilai akurasi yang relatif lebih tinggi dibanding \textit{multi token}.
  \item Relasi semantik lain yang paling banyak dijumpai pada pair yang salah dengan kategori False Relation adalah relasi synonym. 
  \item \textit{Pair} dengan kategori benar instance-class banyak yang merupakan \textit{multi token}, dengan kata hyponym merupakan Proper Noun. 
  \item Beberapa pair masih dapat dianggap benar walau dihilangkan token katanya, seperti (ballantine's, merek wiski) lebih tepat ditulis sebagai (ballantine's,wiski)
  \item Beberapa pair merupakan Noun Phrase seperti (rikard nordraak,komposer norwegia) dimana kata 'komposer' saja sebenarnya sudah cukup. Tambahan kata 'norwegia' dapat menambah informasi relasi antar kata seperti Rikard Nordraak berasal dari Norwegia.
  \item Beberapa pair memiliki token berlebih yang tidak dibutuhkan, seperti 'pesepak bola asal' yang kelebihan kata 'asal' atau 'aslinya paralegal' yang kelebihan kata 'aslinya'. Hal ini terjadi akibat upaya pembentukan \textit{multi token}. 
  \item Beberapa pair merupakan kata fiksi seperti nama karakter suatu cerita.
\end{itemize}

% 5.6
%-----------------------------------------------------------------------------%
\section{Hasil Evaluasi Pattern}
%-----------------------------------------------------------------------------%
Evaluasi \textit{pattern} dilakukan secara manual oleh dua anotator yang ahli dibidang linguistik. Seperti sudah dijelaskan sebelumnya, selain memberi penilaian kepada \textit{pattern} yang dihasilkan oleh sistem, anotator juga membuat \textit{pattern} manual untuk dibandingkan. Berikut adalah hasil \textit{pattern} yang dibuat manual dan evaluasi \textit{pattern} yang dibentuk sistem.

%-----------------------------------------------------------------------------%
\subsection{Pattern Buatan Manual}
%-----------------------------------------------------------------------------%
Lampiran 1 menampilkan daftar \textit{pattern} leksikal yang dibentuk secara manual oleh anotator. Anotator mengamati kalimat-kalimat di dalam teks dokumen yang mengandung pasangan kata hipernim-hiponim kemudian membuat \textit{pattern} yang dapat merepresentasikan relasi. Terdapat total 67 \textit{pattern} leksikal manual yang dibentuk oleh anotator. Tidak seperti \textit{pattern} sistem dimana satu \textit{pattern} hanya dapat mengandung tepat satu pasang \textit{tag} hipernim-hiponim, beberapa \textit{pattern} manual mengandung lebih dari satu \textit{tag} hiponim.

Selanjutnya, \textit{pattern} sistem dibandingkan dengan \textit{pattern} manual tersebut. Satu \textit{pattern} dapat dikategorikan sebagai \textit{exact match}, \textit{partial match}, atau \textit{no match}. Tabel \ref{table:psis-kategori} memperlihatkan jumlah \textit{pattern} yang masuk ke dalam setiap kategori untuk setiap iterasi.

\begin{table}
  \centering
  \caption{Kategori \textit{pattern} sistem dibandingkan dengan \textit{pattern} manual}
  \label{table:psis-kategori}
  \begin{tabular}{|c|c|c|c|}
  \hline
  Percobaan & Exact Match & Partial Match & No Match \\ \hline
  1 & 10 & 114 & 301 \\ \hline
  2 & 10 & 367 & 465 \\ \hline
  3 & 8 & 132 & 297 \\ \hline 
  \end{tabular} 
\end{table}

Jika dibandingkan dengan total \textit{pattern} yang dihasilkan sistem, jumlah \textit{pattern} buatan manual memang jauh lebih sedikit. Tabel diatas memperlihatkan bahwa \textit{pattern} yang tergolong \textit{exact match} masih sangat sedikit. Pada eksperimen 2 dan 3, walau total \textit{pattern} yang dihasilkan berbeda, jumlah \textit{pattern} yang merupakan \textit{exact match} adalah sama. Hal ini berarti semakin banyak \textit{pattern} yang dihasilkan belum tentu semakin baik. Kemudian, dari keseluruhan \textit{pattern}, yang tergolong ke dalam \textit{no match} hampir selalu mendominasi. Hal tersebut memperlihatkan bahwa  \textit{pattern} oleh sistem masih terlalu banyak yang hanya kebetulan terbentuk.

%-----------------------------------------------------------------------------%
\subsection{Hasil Anotasi Pattern}
%-----------------------------------------------------------------------------%
Dua anotator memberi penilaian terhadap 200 \textit{pattern} terbaik yang dihasilkan pada eksperimen ke-1. Anotasi hanya dilakukan pada eksperimen ke-1 karena banyak variasi \textit{pattern} yang dihasilkan antar eksperimen mirip. Selain itu, keterbatasan waktu dan tenaga juga menjadi salah satu alasan. Tabel \ref{table:p-anotasi-benar} menampilkan perbandingan hasil anotasi antar anotator jika dilihat dari dimensi jumlah \textit{pair} yang dihasilkan benar, sementara tabel \ref{table:p-anotasi-salah} memaparkan perbandingan jika dilihat dari dimensi jumlah \textit{pair} yang dihasilkan salah.

\begin{table}
  \centering
  \caption{Perbandingan hasil anotasi berdasarkan jumlah \textit{pair} benar}
  \label{table:p-anotasi-benar}
  \begin{tabular}{|c|c|c|c|}
  \hline
  Jumlah Benar & \multicolumn{3}{c}{ anotator 1 } \\ \hline
  anotator 2 & kategori 1 & kategori 2 & kategori 3 \\ \hline
  kategori 1 & 83 & 5 & 3 \\ \hline
  kategori 2 & 17 & 4 & 27 \\ \hline
  kategori 3 & 10 & 1 & 50 \\ \hline
  \end{tabular} 
\end{table}

\noindent Jika dilihat dari jumlah \textit{pair} benar yang mungkin terekstrak oleh \textit{pattern} tersebut, masih banyak \textit{pattern} yang kualitasnya buruk (kategori 1).

\begin{table}
  \centering
  \caption{Perbandingan hasil anotasi berdasarkan jumlah \textit{pair} salah}
  \label{table:p-anotasi-salah}
  \begin{tabular}{|c|c|c|c|}
  \hline
  Jumlah Salah & \multicolumn{3}{c}{ anotator 1 } \\ \hline
  anotator 2 & kategori 1 & kategori 2 & kategori 3 \\ \hline
  kategori 1 & 57 & 1 & 2 \\ \hline
  kategori 2 & 33 & 1 & 8 \\ \hline
  kategori 3 & 74 & 2 & 22 \\ \hline
  \end{tabular} 
\end{table}

Dari tabel \ref{table:p-anotasi-salah}, sangat banya ditemukan data dimana anotator pertama menganggap \textit{pattern} menghasilkan sedikit \textit{pair} salah (kategori 1) namun anotator dua menilai \textit{pattern} menghasilkan banyak \textit{pair} salah (kategori 3).

Selanjutnya tabel 
\begin{table}
  \centering
  \caption{Hasil anotasi \textit{pattern} yang digunakan untuk ekstraksi \textit{pair}}
  \label{table:anotasi-p-used}
  \begin{tabular}{ccccc} 
  \hline
  \multirow{2}{*}{ Pattern } & \multicolumn{2}{c}{ anotator 1 } & \multicolumn{2}{c}{ anotator 2 } \\ \hline
  & jumlah benar & jumlah salah & jumlah benar & jumlah salah \\ \hline
  <hypernym> atau <hyponym> & 1 & 3 & 1 & 2 \\ \hline
  <hypernym> seperti <hyponym> & 3 & 1 & 3 & 1 \\ \hline
  <hypernym> termasuk <hyponym> & 3 & 1 & 1 & 3 \\ \hline
  <hyponym> adalah <hypernym> yang & 3 & 1 & 3 & 1 \\ \hline
  <hyponym> adalah sebuah <hypernym> & 3 & 1 & 3 & 1 \\ \hline
  <hyponym> menjadi <hypernym> & 3 & 1 & 2 & 2 \\ \hline
  <hyponym> merupakan <hypernym> & 3 & 1 & 3 & 1 \\ \hline
  <hyponym> merupakan <hypernym> yang & 3 & 1 & 3 & 1 \\ \hline
  <start> <hyponym> adalah <hypernym> & 3 & 1 & 3 & 1 \\ \hline
  \end{tabular} 
\end{table}

%-----------------------------------------------------------------------------%
\subsection{Tingkat Persetujuan Anotator Pattern}
%-----------------------------------------------------------------------------%
Dari data pada tabel \ref{table:p-anotasi-benar} dan \ref{table:p-anotasi-salah}, dapat dihitung tingkat persetujuan antar anotator. Perhitungan nilai persetujuan berdasarkan Cohen's Kappa karena jumlah anotator tepat dua. Tabel menampilkan nilai Kappa berdasarkan jumlah \textit{pair} terbentuk benar maupun salah.

\begin{table}
  \centering
  \caption{Cohen's Kappa untuk setiap dimensi}
  \label{table:p-kappa}
  \begin{tabular}{|c|c|c|}
  \hline
  & Jumlah benar & Jumlah Salah \\ \hline
  $P_o$ & 0.685 & 0.4 \\ \hline
  $p_{kategori1}$ & 0.25 & 0.25 \\ \hline
  $p_{kategori2}$ & 0.012 & 0.0042 \\ \hline
  $p_{kategori3}$ & 0.12 & 0.078 \\ \hline
  $P_e$ & 0.38 & 0.33 \\ \hline
  $kappa$ & 0.49 & 0.11 \\ \hline 
  \end{tabular} 
\end{table}

Jika dilihat dari jumlah \textit{pair} yang terbentuk benar, maka tingkat persetujuan antar anotator tergolong {moderate agreemnet}. Sementara jika dilihat dari jumlah \textit{pair} yang terbentuk salah, tingkat persetujuan adalah \textit{slight agreement}. Hal tersebut menunjukan bahwa sangat sulit bagi anotator untuk membayangkan apakah satu \textit{pattern} dapat menghasilkan \textit{pair} salah.

%-----------------------------------------------------------------------------%
\subsection{Analisis Pattern}
%-----------------------------------------------------------------------------%
Ekstraksi \textit{pattern} dibatasi dengan beberapa aturan yang diharapkan dapat menghasilkan \textit{pattern} dengan kualitasi baik. Sistem sudah dapat secara otomatis menghasilkan \textit{pattern} yang tergolong baik untuk merepresentasikan relasi semantik hipernim-hiponim seperti \textit{<hyponym> adalah <hypernym>}, \textit{<hyponym> adalah <hypernym> yang}, \textit{<hyponym> merupakan <hypernym>} dan lainnya. Namun, masih banyak kekurangan dari \textit{pattern} yang dihasilkan. Berikut adalah beberapa kekurangan \textit{pattern} yang terbentuk serta analisa mengapa hal tersebut dapat terjadi.

\begin{itemize}
  \item Belum dapat membentuk \textit{pattern} yang mengandung lebih dari satu \textit{tag hyponym}. \\
  Banyak kalimat yang mengandung lebih dari satu kata \textit{hyponym}, namun saat ini \textit{pattern} yang terbentuk belum dapat membuatnya. Hal ini karena pada masa awal implementasi, ditemukan beberapa kalimat yang mengandung dua pasangan kata relasi berbeda dalam satu kalimat. Untuk mencegah ambiguitas dibatasi satu kalimat hanya akan di-\textit{tag} dengan satu pasang kata relasi. Kedepannya, hal ini dapat diatasi dengan mengubah batasan sehingga dalam satu kalimat boleh di-\textit{tag} dengan lebih dari satu pasang kata relasi asal kata hipernimnya sama. 
  \item Beberapa \textit{pattern} yang kurang baik secara semantik. \\
  Beberapa \textit{pattern} yang dihasilkan sistem seperti \textit{<hypernym> dan <hyponym>} dan  \textit{<hypernym> atau <hyponym>} kurang cocok jika digunakan dalam \textit{pattern matching}. Hal ini disebabkan kata 'dan' dan 'atau' memuat relasi yang bersifat simetris. Sementara hipernim-hiponim merupakan relasi yang hanya memiliki sifat transitif.
  \item \textit{Pattern} yang tergolong dalam lebih dari satu relasi. \\
  Dalam beberapa penelitian lain, satu \textit{pattern} bisa merepresentasikan beberapa relasi semantik. Terutama \textit{pattern} yang berukuran pendek seperti yang dihasilkan sistem. Penelitian ini hanya fokus pada relasi semantik hipernim-hiponim sehingga tidak adanya pembanding antar \textit{pattern} yang merepresentasikan relasi semantik lain. Hal ini menjadi hambatan dalam menentukan apakah \textit{pattern} dapat dikatakan baik.
  % TODO: tambahin referensi ttg ini !!
  \item \textit{Pattern} yang tidak menghasilkan \textit{pair} apapun. \\
  Penggunaan korpus yang sedikit berbeda antara proses pembentukan \textit{pattern} dan proses ekstraksi \textit{pair} serta upaya pembentukan \textit{multiword} membuat sedikit kejanggalan dari hasil \textit{pattern} yang terbentuk. Setiap \textit{pattern} yang terbentuk seharusnya dapat mengekstrak minimal \textit{pair} yang membentuk \textit{pattern} tersebut. Namun, beberapa \textit{pattern} hasil sistem seperti \textit{<hypernym> yunani <hyponym>} tidak mengekstrak \textit{pair} setelah dijalankan dalam proses \textit{pattern matching}. Hal ini disebabkan oleh upaya pembentukan \textit{multi word} yang memanfaatkan korpus Wikipedia dengan \textit{pos tag} membuat beberapa kata bergabung membentuk suatu \textit{multi word}, \textit{proper noun}, atau \textit{noun phrase}. Sementara pada pembentukan \textit{pattern}, batasan tersebut tidak diterapkan.
\end{itemize}

Cara membentuk \textit{pattern} leksikal yang baik masih perlu diteliti lebih lanjut. Proses lain seperti generalisasi \textit{pattern} perlu diimplementasi agar satu \textit{pattern} mengandung informasi yang lebih kaya.

% 5.7
%-----------------------------------------------------------------------------%
\section{Evaluasi Skenario Eksperimen}
%-----------------------------------------------------------------------------%
Dari tiga eksperimen yang telah dilakukan, eksperimen 3 memberi nilai akurasi terbaik (Tabel \ref{table:akurasi-all}). Hal tersebut membuktikan bahwa \textit{pair} yang hanya \textit{single token} lebih tinggi kemunckinan benar dibanding \textit{pair} yang \textit{multi token}. Eksperimen tersebut juga memebuktikan bahwa metode pembentuk \textit{multi token} yang sangat ini digunakan masih perlu peningkatan.

Kemudian, dari hasil perbandingan eksperimen ke-1 dan eksperimen ke-2, menurunkan nilai \textit{threshold} tidak selalu memperburuk kualitas \textit{pair} yang dihasilkan. Akurasi \textit{pair} tergantuk dari \textit{pattern} yang membentuk \textit{pair} tersebut. Jika banyak \textit{pattern} baik yang mengekstrak \textit{pair}, maka besar kemungkinan \textit{pair} tersebut juga berkualitas baik.

Dalam penelitian ini, hasil masih sangat tergantung dengan pendefinisian segala parameter \textit{semi-supervised} yang digunakan. Dari proses pembentukan \textit{seed} dengan menerapkan sejumlah filterisasi, proses pengurutan \textit{pattern}, pemilihan banyak \textit{pattern} yang digunakan, pembentukan \textit{multi token}, pembobotan \textit{pair} yang terekstrak, filterisasi \textit{pair}, serta \textit{stopping condition} masih banyak yang didefinisikan secara manual berdasarkan hasil pengamatan kualitatif. Penelitian ini masih banyak memerlukan pengembangan lebih lanjut agar dapat memberikan hasil yang jauh lebih baik. 
