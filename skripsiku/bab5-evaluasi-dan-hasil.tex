%-----------------------------------------------------------------------------%
\chapter{\babLima}
%-----------------------------------------------------------------------------%
Bab ini menjelaskan mengenai hasil untuk setiap tahapan penelitian, deskripsi percobaan yang telah dilakukan, serta evaluasi dan hasil terkait percobaan tersebut.

%-----------------------------------------------------------------------------%
\section{Pengumpulan Data Wikipedia}
%-----------------------------------------------------------------------------%
Korpus teks dokumen utama dalam penelitian ini adalah artikel Wikipedia. Korpus diunduh dari situs Wikimedia. Berkas yang digunakan adalah idwiki-20170201-pages-article-multistream.xml.bz2, diunduh pada 20 Februari 2017. Berkas tersebut berukuran 398.6 MB dan merupakan data artikel Wikipedia Bahasa Indonesia hingga tanggal 1 Februari 2017. Setelah di-\textit{extract}, ukuran asli berkas XML tersebut adalah 1.9 GB. Berkas tersebut mengandung seluruh \textit{tag} identitas Wikipedia dan ditulis dalam format metadata WIkipedia. 


%-----------------------------------------------------------------------------%
\section{Hasil Pengolahan Data Wikipedia}
%-----------------------------------------------------------------------------%
Sebelum digunakan sebagai korpus masukan untuk proses \textit{pattern matching} dan \textit{extraction}, data Wikipedia terlebih dahulu diproses.

%-----------------------------------------------------------------------------%
\subsection{Ekstraksi Teks}
%-----------------------------------------------------------------------------%
Proses ekstraksi teks dilakukan karena tidak seluruh bagian teks digunakan sebagai data penelitian. Data wikipedia yang ingin digunakan hanya isi artikel. Korpus Wikipedia diekstrak menggunakan WikiExtractor untuk menghilangkan \textit{tag} yang tidak digunakan. Selain itu, artikel Wikipedia juga perlu dibersihkan dari simbol-simbol metadata. Setelah dilakukan proses ini, total ukuran berkasi Wikipedia menjadi 424 MB.

%-----------------------------------------------------------------------------%
\subsection{Pembentukan Kalimat}
%-----------------------------------------------------------------------------%
Data hasil ekstraksi memisahkan satu paragraf untuk setiap baris, sementara format yang diinginkan adalah satu baris merepresentasikan satu kalimat. Digunakan program \textit{sentence splitter} untuk memisahkan kalimat dalam berkas Wikipedia. Kemudian, dilanjutkan ke dalam proses \textit{rule-based formatter} sehingga memberi hasil kalimat yang sudah sesuai format yang ditentukan. Proses tersebut menghasilkan 3.696.339 kalimat dengan total ukuran berkas 431 MB. Berkas ini digunakan sebagai masukan proses \textit{pattern extraction.} 

%-----------------------------------------------------------------------------%
\subsection{Hasil POS Tagging Kalimat}
%-----------------------------------------------------------------------------%
Kalimat-kalimat yang telah dibentuk, dimasukan ke dalam program POS Tagger, sehingga dihasilkan kalimat yang setiap tokennya memiliki \textit{tag} berdasarkan kelas kata. Total ukuran berkas adalah 623 MB. Berkas ini digunakan sebagai masukan proses \textit{pattern matching}. 

%-----------------------------------------------------------------------------%
\subsection{Pemodelan Word Embedding}
%-----------------------------------------------------------------------------%
Pembuatan model \textit{word embedding} menggunakan korpus Wikipedia yang sudah berbentuk kalimat. Proses ini menghasilkan tiga berkas model (tabel \ref{table:modelWE}). Berkas hasil pemodelan berukuran besar karena menggunakan seluruh kalimat yang ada dalam korpus Wikipedia.
\begin{table}
  \centering
    \caption{Berkas model \textit{word embedding}}
    \label{table:modelWE}
    \begin{tabular}{|c|c|c|}
      \hline
        Nama Berkas                       & Jenis Berkas & Ukuran Berkas \\ \hline
        model-word2vec                    & File         & 306.1 MB      \\ \hline
        model-word2vec.syn0.npy           & NPY File     & 1.8 GB        \\ \hline
        model-word2vec.syn1neg.npy        & NPY File     & 1.8 GB        \\ \hline
    \end{tabular}
\end{table}

%-----------------------------------------------------------------------------%
\section{Pengumpulan Seed}
%-----------------------------------------------------------------------------%
Pasangan kata relasi \textit{hypernym-hyponym} awal diambil dari korpus yang dimiliki oleh WordNet Bahasa. Setelah melalui proses pembentukan \textit{seed}, berikut adalah jumlah \textit{seed} yang dihasilkan berdasarkan jenis filterisasi. Kedua tipe \textit{seed} tersebut digunakan proses pembentukan pattern pada iterasi pertama. 
\begin{table}
  \centering
    \caption{Jumlah \textit{seed} hasil ekstraksi}
      \label{table:jumlahSeed}
      \begin{tabular}{|c|c|c|}
        \hline
          No & Jenis Filterisasi & Jumlah Seed \\ \hline
          1. & lema sama         & 8.985       \\ \hline
          2. & \textit{strict}   & 8.602       \\ \hline
    \end{tabular}
\end{table}

Walau sudah melakukan proses filterisasi sebagai upaya mengurangi \textit{error} dan ambiguitas yang terjadi pada proses pemenentukan \textit{seed} serta untuk meningkatkan kulatias \textit{seed}, masih ada hambatan yang belum dapat diatasi dalam penelitian ini. Beberapa kelemahan dari \textit{seed} awal yang dihasilkan adalah sebagai berikut.
\begin{itemize}
  \item \textit{Seed} yang mengandung kata bukan Bahasa Indonesia. Korpus yang ingin dibuat berdomain Bahasa Indonesia, namun \textit{seed} yang dihasilkan mengandung Bahasa Melayu atau Bahasa Indonesia lama. Beberapa kata bukan Bahasa Indonesia yang dihasilkan adalah 'had', 'bonjol', dan 'cecok'.
  \item Kesalahan semantik \textit{synset} dan lema Bahasa Indonesia. Beberapa \textit{synset} nltk memiliki lema Bahasa Indonesia yang kurang sesuai jika dilihat secara semantik. Sebagai contoh \textit{Synset('scholar.n.01')} dengan lemma Bahasa Indonesia \textit{{'buku\_harian', 'pelajar'}}. Dalam Bahasa Indonesia, 'buku\_harian' memiliki makna yang berbeda dengan 'pelajar'.
  \item Kesalahan lema Bahasa Indonesia untuk suatu \textit{synset} menyebabkan dihasilkannya \textit{seed} yang jika dievaluasi kualitatif oleh manusia dirasa kurang tepat. Contoh \textit{seed} yang tidak baik adalah dari pemetaan \textit{('sejarawan', Synset('historian.n.01')) => (['buku\_harian', 'pelajar'], [Synset('scholar.n.01')])} dihasilkan \textit{seed} \textit{(sejarawan,buku harian)} dan \textit{(sejarawan,pelajar)}. \textit{Seed} (sejarawan,buku harian) adalah salah.
\end{itemize}


%-----------------------------------------------------------------------------%
\section{Pembentukan Pattern untuk Iterasi Pertama}
%-----------------------------------------------------------------------------%
Iterasi pertama untuk seluruh percobaan selalu menggunakan \textit{seed} diatas, sehingga hasil untuk proses pembentukan \textit{pattern} sama. Berikut adalah detil hasil untuk \textit{sentence tagging} dan \textit{pattern extraction} pada iterasi pertama.

%-----------------------------------------------------------------------------%
\subsection{Sentence Tagging dengan Seed}
%-----------------------------------------------------------------------------%
Dari kedua tipe \textit{seed}, masing-masing digunakan untuk membentuk korpus kalimat yang memiliki \textit{tag} relasi \textit{hypernym-hyponym}. Berikut adalah hasil proses sentence tagging menggunakan kedua tipe \textit{seed}.
\begin{table}
  \centering
  \caption{Hasil \textit{sentence tagging} dengan \textit{seed}}
  \label{table:sentencetagging1}
  \begin{tabular}{|c|c|c|c|}
    \hline
      No. & Jenis Seed                        & Jumlah kalimat di-tag & Ukuran berkas \\ \hline
      1.  & Seed dengan filterisasi lema sama & 69.574                & 14 MB         \\ \hline
      2.  & Seed dengan filterisasi strict    & 64.718                & 13 MB         \\ \hline
  \end{tabular}
\end{table}

%-----------------------------------------------------------------------------%
\subsection{Hasil Pattern Extraction}
%-----------------------------------------------------------------------------%
Kedua korpus kalimat yang masing-masing menghasilkan daftar \textit{pattern}. Kedua daftar \textit{pattern} digabung dan diurutkan sesuai bobot. Dari 106 \textit{pattern} yang dihasilkan pada iterasi pertama, tabel \ref{table:pattern1} memaparkan lima \textit{pattern} terbaik yang digunakan untuk proses \textit{pattern matching}. Kelima \textit{pattern} tersebut dibentuk dari total 104 \textit{seed} yang langsung masuk ke korpus pasangan kata relasi.
\begin{table}
  \centering
  \caption{Pattern terbaik iterasi pertama}
  \label{table:pattern1}
  \begin{tabular}{|c|}
    \hline
      start <hyponym> adalah <hypernym> \\
      <hyponym> merupakan <hypernym> \\ 
      <hyponym> adalah <hypernym> yang \\
      <hypernym> seperti <hyponym> dan \\
      <hypernym> termasuk <hyponym> \\ \hline
  \end{tabular}
\end{table}


%-----------------------------------------------------------------------------%
\section{Hasil Eksperimen}
%-----------------------------------------------------------------------------%
Bagian ini akan memaparkan penjelasan parameter untuk setiap eksperimen yang telah dilakukan dan hasil dari eksperimen tersebut. 

%-----------------------------------------------------------------------------%
\subsection{Eksperimen 1}
%-----------------------------------------------------------------------------%
Eksperimen pertama dilakukan dengan nilai \textit{threshold} untuk filterisasi \textit{pair} baru sebesar 0.6. Tabel \ref{table:eksp1} memaparkan jumlah \textit{pattern}, \textit{pattern} baru, total \textit{pair} hasil ekstraksi, dan total korpus pasangan kata relasi yang dihasilkan untuk setiap iterasi.
\begin{table}
  \centering
  \caption{Hasil Eksperimen 1}
  \label{table:eksp1}
  \begin{tabular}{|c|m{14em}|m{4em}|m{4em}|m{4em}|}
    \hline
      No. & Pattern Baru Terpilih & Total Pattern & Total Pair yang Dihasilkan & Korpus Pasangan Kata Relasi \\ \hline
      1.  &                                            & 106  &  82409   &  939 \\ \hline
      2.  & <start> <hyponym> merupakan <hypernym>     & 347  &  104389  &  1255 \\ \hline
      3.  & <hyponym> merupakan <hypernym> yang        & 395  &  108397  &  1430 \\ \hline
      4.  & <hypernym> atau <hyponym>                  & 425  &  108542  &  1432 \\ \hline
  \end{tabular}
\end{table}

\noindent Jumlah \textit{pair} yang nilainya melebihi \textit{threshold} dapat dikatakan sedikit jika dibandingkan dengan seluruh \textit{pair} yang terekstrak. 

%-----------------------------------------------------------------------------%
\subsection{Eksperimen 2}
%-----------------------------------------------------------------------------%
Melihat sedikitnya jumlah pattern yang dihasilkan dari eksperimen pertama, maka diputuskan untuk menurunkan sedikit nilai \textit{threshold}. Nilai \textit{threshold} untuk filterisasi \textit{pair} baru menjadi sebesar 0.55. Tabel \ref{table:eksp2} memaparkan hasil dari eksperimen kedua.
\begin{table}
  \centering
  \caption{Hasil Eksperimen 2}
  \label{table:eksp2}
  \begin{tabular}{|c|m{14em}|m{4em}|m{4em}|m{4em}|}
    \hline
      No. & Pattern Baru Terpilih & Total Pattern & Total Pair yang Dihasilkan & Korpus Pasangan Kata Relasi \\ \hline
      1.  &                                            & 106  &  82409   &  2193  \\ \hline
      2.  & <start> <hyponym> merupakan <hypernym>     & 699  &  104389  &  3071 \\ \hline
      3.  & <hyponym> merupakan <hypernym> yang        & 791  &  108397  &  3484 \\ \hline
      4.  & <hyponym> menjadi <hypernym>               & 842  &  108737  &  3493 \\ \hline
  \end{tabular}
\end{table}

\noindent Jika dibandingkan dengan eksperiment pertama, jumlah \textit{pair} yang nilainya melebihi \textit{threshold} bertambah hingga dua kali dari sebelumnya walaupun selisih nilai \textit{threshold} hanya 0.05. Ini berarti banyak \textit{pair} yang nilai bobotnya dekat dengan nilai batas tersebut. Jika dilihat dari total \textit{pair} yang diekstrak, hingga iterasi ke-3 jumlahnya sama karena \textit{pattern} yang digunakan juga sama. Pada iterasi ke-4 dimana \textit{pattern} eksperimen ke-1 dan ke-2 diekstrak total \textit{pair} berbeda.

%-----------------------------------------------------------------------------%
\subsection{Eksperimen 3}
%-----------------------------------------------------------------------------%
Setelah menganalisis secara kualitatif \textit{pair} yang dihasilkan, banyak \textit{pair} yang kata \textit{hypernym}-nya adalah \textit{class} sementara kata \textit{hyponym}-nya adalah \textit{instance}. Untuk relasi semantik \textit{hypernym-hyponym} yang lebih umum, hubungan antar kata yang lebih diinginkan adalah \textit{class-class}. Untuk pasangan kata yang bertipe \textit{class-class} lebih banyak dijumpai jika kedua kata merupakan \textit{single word}. Selain it, pembentukan \textit{multi word} yang dilakukan belum sepenuhnya baik karena masih banyak kata yang kelebihan token kata. Untuk itu, dicoba eksperimen yang hanya mengekstrak \textit{pair} yang kedua katanya adalah \textit{single word}. Untuk eksperimen ini, menggunakan \textit{threshold} sebesar 0.55.
\begin{table}
  \centering
  \caption{Hasil Eksperimen 3}
  \label{table:eksp3}
  \begin{tabular}{|c|m{14em}|m{4em}|m{4em}|m{4em}|}
    \hline
      No. & Pattern Baru Terpilih & Total Pattern & Total Pair yang Dihasilkan & Korpus Pasangan Kata Relasi \\ \hline
      1.  &                                            & 106  &  10267   &  438  \\ \hline
      2.  & <hyponym> adalah sebuah <hypernym>         & 437  &  11262   &  459 \\ \hline
  \end{tabular}
\end{table}

\noindent Untuk \textit{pair} yang hanya \textit{single word} sudah pasti jumlahnya jauh lebih sedikit dibanding jika dibanding keseluruhan. Jika dibandingkan dengan dua eksperimen sebelumnya, total \textit{pattern} yang dihasilkan juga jauh lebih sedikit. Dapat dikatakan bahwa jumlah pasangan kata yang digunakan mempengaruhi jumlah \textit{pattern} yang dihasilkan, dimana semakin banyak jumlah pasangan kata relasi maka semakin banyak pula \textit{pattern} yang dihasilkan.


%-----------------------------------------------------------------------------%
\section{Hasil Evaluasi Pattern}
%-----------------------------------------------------------------------------%
Evaluasi \textit{pattern} dilakukan secara manual oleh dua anotator yang ahli dibidang linguistik. Seperti sudah dijelaskan sebelumnya, selain memberi penilaian kepada \textit{pattern} yang dihasilkan oleh sistem, anotator juga membuat \textit{pattern} manual untuk dibandingkan. Berikut adalah hasil \textit{pattern} yang dibuat manual dan evaluasi \textit{pattern} yang dibentuk sistem.

%-----------------------------------------------------------------------------%
\subsection{Pattern Buatan Manual}
%-----------------------------------------------------------------------------%
Berikut adalah daftar \textit{pattern} leksikal yang dibentuk secara manual oleh anotator. Anotator mengamati kalimat-kalimat di dalam teks dokumen yang mengandung pasangan kata \textit{hypernym-hyponym}, kemudian membuat \textit{patter} yang dapat merepresentasikan relasi. Daftar \textit{pattern} manual dapat dilihat pada lampiran 1. Tidak seperti \textit{pattern} yang dihasilkan oleh sistem dimana satu \textit{pattern} hanya dapat mengandung satu pasang \textit{tag hypernym-hyponym}, beberapa \textit{pattern} manual mengandung lebih dari satu \textit{tag hyponym}.

\noindent \textbf{\textit{on-going}}: perbandingan pattern manual dan pattern hasil sistem

%-----------------------------------------------------------------------------%
\subsection{Hasil Anotasi Pattern}
%-----------------------------------------------------------------------------%
Dua anotator memberi penilaian terhadap sejumlah \textit{pattern} yang sama. 

\noindent \textbf{\textit{on-going}}: hasil anotasi pattern, evaluasi dan analisis.


%-----------------------------------------------------------------------------%
\subsection{Analisis Pattern Hasil Sistem}
%-----------------------------------------------------------------------------%
Ekstraksi \textit{pattern} dibatasi dengan beberapa aturan yang diharapkan dapat menghasilkan \textit{pattern} dengan kualitasi baik. Sistem sudah dapat secara otomatis menghasilkan \textit{pattern} yang tergolong baik untuk merepresentasikan relasi semantik \textit{hypernym-hyponym} seperti \textit{<hyponym> adalah <hypernym>}, \textit{<hyponym> adalah <hypernym> yang}, \textit{<hyponym> merupakan <hypernym>} dan lainnya. Namun, masih banyak kekurangan dari \textit{pattern} yang dihasilkan. Berikut adalah beberapa kekurangan \textit{pattern} yang terbentuk serta analisa mengapa hal tersebut dapat terjadi.

\begin{itemize}
  \item Belum dapat membentuk \textit{pattern} yang mengandung lebih dari satu \textit{tag hyponym}. \\
  Banyak kalimat yang mengandung lebih dari satu kata \textit{hyponym}, namun saat ini \textit{pattern} yang terbentuk belum dapat membuatnya. Hal ini karena pada masa awal implementasi, ditemukan beberapa kalimat yang mengandung dua pasangan kata relasi berbeda dalam satu kalimat. Untuk mencegah ambiguitas dibatasi satu kalimat hanya akan di-\textit{tag} dengan satu pasang kata relasi. Kedepannya, hal ini dapat diatasi dengan mengubah batasan sehingga dalam satu kalimat boleh di-\textit{tag} dengan lebih dari satu pasang kata relasi asal kata \textit{hypernym}-nya sama. 
  \item Beberapa \textit{pattern} yang kurang baik secara semantik. \\
  Beberapa \textit{pattern} yang dihasilkan sistem seperti \textit{<hypernym> dan <hyponym>} dan  \textit{<hypernym> atau <hyponym>} kurang cocok jika digunakan dalam \textit{pattern matching}. Hal ini disebabkan kata 'dan' dan 'atau' memuat relasi yang bersifat simetris. Sementara \textit{hypernym-hyponym} merupakan relasi yang hanya memiliki sifat transitif.
  \item \textit{Pattern} yang tergolong dalam lebih dari satu relasi. \\
  Dalam beberapa penelitian lain, satu \textit{pattern} bisa merepresentasikan beberapa relasi semantik. Terutama \textit{pattern} yang berukuran pendek seperti yang dihasilkan sistem. Penelitian ini hanya fokus pada relasi semantik \textit{hypernym-hyponym} sehingga tidak adanya pembanding antar \textit{pattern} yang merepresentasikan relasi semantik lain. Hal ini menjadi hambatan dalam menentukan apakah \textit{pattern} dapat dikatakan baik.
  % TODO: tambahin referensi ttg ini !!
  \item \textit{Pattern} yang tidak menghasilkan \textit{pair} apapun. \\
  Penggunaan korpus yang sedikit berbeda antara proses pembentukan \textit{pattern} dan proses ekstraksi \textit{pair} serta upaya pembentukan \textit{multiword} membuat sedikit kejanggalan dari hasil \textit{pattern} yang terbentuk. Setiap \textit{pattern} yang terbentuk seharusnya dapat mengekstrak minimal \textit{pair} yang membentuk \textit{pattern} tersebut. Namun, beberapa \textit{pattern} hasil sistem seperti \textit{<hypernym> yunani <hyponym>} tidak mengekstrak \textit{pair} setelah dijalankan dalam proses \textit{pattern matching}. Hal ini disebabkan oleh upaya pembentukan \textit{multi word} yang memanfaatkan korpus Wikipedia dengan \textit{pos tag} membuat beberapa kata bergabung membentuk suatu \textit{multi word}, \textit{proper noun}, atau \textit{noun phrase}. Sementara pada pembentukan \textit{pattern}, batasan tersebut tidak diterapkan.
\end{itemize}

Cara membentuk \textit{pattern} leksikal yang baik masih perlu diteliti lebih lanjut. Proses lain seperti generalisasi \textit{pattern} perlu diimplementasi agar satu \textit{pattern} mengandung informasi yang lebih kaya.


%-----------------------------------------------------------------------------%
\section{Hasil Anotasi Pair}
%-----------------------------------------------------------------------------%
Anotasi \textit{pair} dilakukan oleh tiga anotator berbeda dengan daftar \textit{pair} yang sama. \textit{Pair} yang digunakan untuk proses anotasi diambil secara acak menggunakan teknik \textit{random sampling}.

\noindent \textbf{\textit{on-going}}: hasil anotasi pair, perbandingan anotasi antar anotator, akurasi pair, evaluasi dan analisis. 
