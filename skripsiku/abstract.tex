%
% Halaman Abstract
%
% @author  Andreas Febrian
% @version 1.00
%

	\chapter*{ABSTRACT}

\vspace*{0.2cm}

\noindent \begin{tabular}{l l p{11.0cm}}
	Name&: & \penulis \\
	Program&: & \programEng \\
	Title&: & \judulInggris \\
\end{tabular} \\ 

\vspace*{0.5cm}

\noindent 
\\ Word relation extraction is one of research branch in Natural Language Processing (NLP) which purpose it to extract words based on the defined relations. Corpus which hold the relationship among words can be used for other future research. Usually, those corpus can be found in one of the famous digital dictionary which is WordNet. Unfortunatelly, the current WordNet Bahasa Indonesia is not perfect and still contains many errors. This research aims to build a semantically related words corpus Bahasa Indonesia automatically. Many other researchs had been done before with various methods. The method that will be used in this research is pattern extraction and pattern matching. The corpus will be build using semi-supervised learning. The relation that is observed is semantic relation hypernym and hyponym and this research use Wikipedia as the datasource. At the end of this research, a corpus which contains 3493 pair of related words has been built. The accuracy of each experiments are above 0.80. Those infomations show that using pattern analysis to extract pair related words from Wikipedia has a potential to gather large sum of data and having good quality.

\vspace*{0.2cm}

\noindent Keywords: \\ 
\noindent  word relation, pattern extraction, pattern matching, semi-supervised, hypernym, hyponym, corpus \\
\newpage
