%
% Halaman Abstract
%
% @author  Andreas Febrian
% @version 1.00
%

\chapter*{ABSTRACT}

\vspace*{0.2cm}

\noindent \begin{tabular}{l l p{11.0cm}}
	Name&: & \penulis \\
	Program&: & \programEng \\
	Title&: & \judulInggris \\
\end{tabular} \\ 

\vspace*{0.5cm}

\noindent 
\\ Word relation extraction is one of research branch in Natural Language Processing (NLP) which purpose is to extract words based on the defined relations. Corpus that holds the relationship among words is used for other future research. For English, those corpus can be found in one of the famous digital dictionary which is WordNet. Unfortunatelly, the current WordNet Bahasa Indonesia is not perfect and still contains errors. This research aims to automatically build a semantically related words corpus for Bahasa Indonesia. Many other researches had been done before with various methods. In this research, pattern analysis aproach was used which consists of pattern extraction and pattern matching. The corpus was build gradually using semi-supervised learning. The relation that was observed is semantic relation hypernym and hyponym and this research used Wikipedia as the datasource. At the end of this research, a corpus which contains 3493 pair of related words had successfully been built. The accuracy for every experiments were above 0.80. Those infomations show that using pattern analysis to extract pair of related words from Wikipedia has a potential to be used for gathering large sum of data and having good quality.

\vspace*{0.2cm}

\noindent Keywords: \\ 
\noindent  word relation, pattern extraction, pattern matching, semi-supervised, hypernym, hyponym, corpus \\
\newpage
