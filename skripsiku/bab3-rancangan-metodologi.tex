%-----------------------------------------------------------------------------%
\chapter{\babTiga}
%-----------------------------------------------------------------------------%
Pada bab ini dipaparkan mengenai rancangan dan tahap-tahap proses ekstraksi relasi semantik, mulai dari pengumpulan seed, proses pattern extraction dan matching, dan evaluasi hingga dihasilkan korpus pasangan relasi kata hypernym-hyponym Bahasa Indonesia. 


%-----------------------------------------------------------------------------%
\section{Spesifikasi Penelitian}
%-----------------------------------------------------------------------------%
Berikut adalah spesifikasi penelitian yang dilaksanakan dan batas-batasanya.
\begin{itemize}
  \item Relasi yang diperhatikan adalah \textit{hypernym-hyponym}.
  \item Kelas kata yang diperhatikan adalah \textit{noun} dan \textit{proper noun}.
  \item Kata yang diekstrak termasuk \textit{multi word}. Contoh: sepak bola, Amerika Serikat, dan lainnya.
\end{itemize}


%-----------------------------------------------------------------------------%
\section{Rancangan Pengembangan Korpus}
%-----------------------------------------------------------------------------%
Pengembangan korpus pasangan kata berelasi yang digunakan pada penelitian ini menggunakan arsitektur yang dapat dilihat pada gambar 3.1. Terdapat dua sumber data utama yang digunakan yaitu WordNet Bahasa untuk pembentukan \textit{seed} dan artikel Wikipedia Bahasa Indonesia sebagai korpus teks. Secara garis besar, terdapat enam tahap yang perlu dilakukan yaitu pembuatan \textit{seed}, preprocessing data Wikipedia,\textit{sentence tagging}, \textit{pattern extraction}, \textit{pattern matching}, dan terakhir adalah evaluasi. Untuk proses \textit{sentence tagging}, \textit{pattern extraction}, dan \textit{pattern matching} dilakukan secara berulang, sesuai dengan metode \textit{bootstrapping}. Berikut adalah penjelasan singkat setiap tahapan.
\begin{enumerate}
  \item Pembentukan \textit{seed} dilakukan untuk mendapatkan pasangan kata berelasi yang digunakan sebagai dasar penelitian. Proses ini memanfaatkan \textit{resource} yang dimiliki WordNet Bahasa.
  \item Artikel Wikipedia diperoleh dalam bentuk \textit{Wikipedia dump} dan perlu diolah sehingga memiliki representasi sesuai dengan format yang diharapkan. Informasi yang diperlukan hanya bagian isi artikel dan kemudian ditulis berdasarkan kalimat untuk setiap baris.
  \item Menggunakan \textit{seed} dan korpus Wikipedia, dilakukan \textit{tagging} pasangan kata berelasi terhadap kalimat-kalimat yang ada. Kalimat yang mengandung pasangan kata berelasi akan ditandai dan disimpan sebagai dasar proses selanjutnya.
  \item Kalimat-kalimat yang sudah di-\textit{tag} dengan pasangan kata berelasi kemudian akan digunakan untuk proses \textit{pattern extraction}. Hasil dari proses ini adalah sejumlah \textit{pattern} leksikal terbaik dari banyak \textit{pattern} yang dihasilkan.
  \item Proses berikutnya adalah \textit{pattern matching} dimana \textit{pattern} hasil ekstraksi dan korpus Wikipedia kembali digunakan untuk membentuk pasangan kata relasi baru (\textit{pair}).
  \item Korpus pasangan relasi kata yang terbentuk digunakan untuk iterasi selanjutnya sesuai dengan metode \textit{bootstrapping}. Proses iterasi dilakukan hingga pasangan kata relasi baru yang dihasilkan jenuh. Terakhir dilakukan evaluasi untuk mengetahui akurasi data yang dihasilkan.
\end{enumerate}
Proses ini diharapkan dapat menghasilkan korpus pasangan kata relasi \textit{hyponym-hypernym} yang berkualitas baik dan berukuran besar. \textit{Pair} yang dihasilkan ditulis dalam bentuk \textit{tuple} $(kata_hyponym;kata_hypernym)$ dengan kedua kata berada dalam kelas kata benda.


%-----------------------------------------------------------------------------%
\section{Pre-processing Data}
%-----------------------------------------------------------------------------%
Proses inti dari penelitian ini, \textit{pattern extraction} dan \textit{matching}, memerlukan dua masukan utama yaitu sejumlah pasangan kata \textit{hypernym-hyponym} dan teks dokumen yang digunakan sebagai korpus. Pasangan kata \textit{hypernym-hyponym} digunakan untuk proses pembentukan \textit{pattern} sementara teks dokumen digunakan untuk memperoleh pasangan kata baru. Dikarenakan belum ada korpus pasangan kata relasi \textit{hypernym-hyponym} Bahasa Indonesia, perlu didefinisikan \textit{seed} yang akan digunakan sebagai dasar pasangan kata \textit{hypernym-hyponym}. Teks dokumen yang digunakan, yaitu Wikipedia, juga memerlukan pemrosesan sebelum digunakan.

%-----------------------------------------------------------------------------%
\subsection{Pre-processing Data Wikipedia}
%-----------------------------------------------------------------------------%
Data Wikipedia yang diperoleh dari Wikimedia \textit{dumps} masih mengandung banyak \textit{tag} yang tidak digunakan pada penelitian ini seperti \textit{tag} \textit{id} dan \textit{revision}. Selain itu simbol-simbol khusus (\textit{markup}). Penelitian ini ingin mengekstrak \textit{pattern} dari \textit{free text}, sehingga format-format khusus tersebut perlu dibersihkan. Setelah data Wikipedia dibersihkan dari simbol-simbol tersebut, langkah selanjutnya adalah merepresentasikan korpus dalam bentuk kalimat. 

Artikel-artikel Wikipedia dibentuk ke dalam format yang telah didefinisikan dengan satu kalimat dipisahkan untuk setiap barisnya. Data tersebut juga digunakan untuk proses \textit{part-of-speech tagging}. Hasil dari \textit{pre-processing} data Wikipedia adalah dua korpus besar yaitu korpus tanpa \textit{pos tag} yang digunakan sebagai masukan \textit{pattern extraction} dan korpus dengan \textit{pos tag} yang digunakan sebagai masukan \textit{pattern matching}. 

%-----------------------------------------------------------------------------%
\subsection{Pembentukan Seed}
%-----------------------------------------------------------------------------%
Pasangan kata relasi \textit{hypernym-hyponym} diambil dari data yang dimilki oleh WordNet Bahasa yang dikembangkan oleh NTU. Pemanfaatan WordNet Bahasa dilatarbelakangi jumlahnya yang lebih besar dibanding Indonesian WordNet (IWN). Relasi semantik antar kata pada WordNet Bahasa memanfaatkan WordNet Princeton versi 3.0. Synset pada WordNet Bahasa dipetakan ke \textit{synset} WordNet Princeton, sehingga relasi kata yang dimiliki oleh WordNet Princeton ikut diwarisi. Alasan lain penggunaan WordNet Bahasa karena telah terintegrasi dengan \textit{tools} nltk sehingga dapat langsung digunakan untuk membentuk \textit{seed} secara mudah.

Seluruh lema Bahasa Indonesia yang dimiliki oleh WordNet tersebut akan dipasangkan dengan lema \textit{hypernym}-nya, sehingga terbentuk relasi biner antara kata yang merupakan \textit{hyponym} dan kata yang merupakan \textit{hypernym}. Format untuk korpus ini mengikut format korpus pasangan kata relasi yang akan digunakan.


%-----------------------------------------------------------------------------%
\section{Pembentukan Pattern}
%-----------------------------------------------------------------------------%
\textit{Pattern} leksikal yang akan digunakan perlu dibentuk secara otomatis oleh sistem. Terdapat dua masukan utama untuk proses ini yaitu pasangan kata relasi \textit{hypernym-hyponym} yang telah diketahui dan korpus dokumen. Pada tahap awal, pasangan kata relasi menggunakan \textit{seed} yang dibentuk menggunakan WordNet Bahasa. Untuk tahap selanjutnya, pasangan kata relasi menggunakan \textit{pair} hasil proses ekstraksi. Terdapat dua proses utama dalam pembentukan \textit{pattern} yaitu \textit{sentence tagging} dan p\textit{attern extraction}.

%-----------------------------------------------------------------------------%
\subsection{Sentence Tagging}
%-----------------------------------------------------------------------------%
\textit{Sentence tagging} adalah proses \textit{intermediate} sebelum dapat membentuk sebuah \textit{pattern} leksikal. Proses ini akan memberi \textit{tag hypernym} dan \textit{hyponym} terhadap suatu kata di dalam dokumen. Masukan untuk proses ini adalah pasangan kata relasi dan korpus Wikipedia tanpa \textit{pos tag}. Untuk setiap kalimat dalam korpus Wikipedia, dicek apakah kalimat tersebut mengandung pasangan kata relasi. Jika mengandung, maka kata dalam kalimat yang merupakan pasangan kata akan di \textit{tag hypernym} dan \textit{hyponym}. Satu kalimat dicocokan dengan seluruh pasangan kata relasi karena terdapat kasus dimana satu kalimat mengandung lebih dari satu pasangan kata relasi. Proses ini menghasilkan kalimat-kalimat Wikipedia yang telah diberi \textit{tag hypernym} atau \textit{hyponym}.

%-----------------------------------------------------------------------------%
\subsection{Pattern Extraction}
%-----------------------------------------------------------------------------%
Setelah didapatkan kumpulan kalimat yang sudah diberi \textit{tag hypernym} dan \textit{hyponym}, proses selanjutnya adalah \textit{pattern extraction}. \textit{Pattern extraction} adalah proses untuk mendapatkan \textit{pattern} leksikal yang kemunculannya sering dalam korpus. Masukan dari proses ini adalah kalimat-kalimat Wikipedia yang telah diberi \textit{tag hypernym} dan \textit{hyponym}. Proses ini akan menghasilkan kumpulan \textit{pattern} unik yang telah terurut berdasarkan bobotnya. Dari \textit{pattern} unik tersebut, akan diambil sejumlah \textit{pattern} terbaik untuk kemudian digunakan sebagai dasar pembentukan \textit{pair} baru.


%-----------------------------------------------------------------------------%
\section{Ekstraksi Pair}
%-----------------------------------------------------------------------------%



%-----------------------------------------------------------------------------%
\section{Cycle Semi-Supervised}
%-----------------------------------------------------------------------------%
Pembelajaran menggunakan pendekatan \textit{semi supervised learning} dilatarbelakangi ketersediaan pasangan kata relasi (seed) berukuran terbatas dan korpus yang belum dianotasi. Ingin didapatkan \textit{pair} baru dari korpus yang belum diantoasi tersebut. Metode \textit{semi supervised} yang diterapkan adalah \textit{Bootstrapping}. Untuk lebih spesifiknya, algoritma \textit{bootstrapping} yang digunakan adalah \textit{Basilisk} dengan modifikasi sesuai kebutuhan. Pemilihan \textit{Basilisk} sebagai metode \textit{Bootrapping} didasari proses ekstraksi \textit{seed} baru yang memanfaatkan \textit{pattern}. Pada penelitian ini, dilakukan modifikasi sesuai kebutuhan. Secara umum, proses \textit{bootstrapping} dapat dikelompokan ke dalam dua tahap yaitu iterasi ke-1 dan iterasik ke-2 hingga n.

Penggunaan \textit{seed} yang berasal dari WordNet Bahasa hanya dilakukan pada iterasi pertama. Pada iterasi ini, pattern yang merupakan hasil dari \textit{seed} dengan lema sama dan \textit{seed strict} digabung dan diurutkan berdasarkan vektor \textit{pattern}. Lima \textit{pattern} terbaik diambil untuk selanjutnya digunakan dalam proses \textit{pattern matching}. \textit{Seed} yang membentuk kelima pattern ini disimpan langsung ke dalam korpus pasangan relasi kata (\textit{pair}). Hal tersebut dapat memfilter \textit{pair} yang dihasilkan akibat \textit{error} dalam \textit{resource} WordNet Bahasa tidak ikut terambil. \textit{Pair} yang memenuhi \textit{threshold} dimasukan ke dalam korpus pasangan kata relasi. 

Pada iterasi ke-2 hingga n, \textit{pair} dalam korpus pasangan relasi kata digunakan sebagai \textit{seed} untuk proses \textit{sentence tagging} dan \textit{pattern extraction}. \textit{Pattern} yang dihasilkan akan digabung dengan seluruh \textit{pattern} lama kemudian di \textit{ranking}. Seperti pada metode \textit{Basilisk}, jumlah \textit{pattern} yang digunakan pada iterasi tersebut akan ditambah. Satu \textit{pattern} terbaik dari hasil pengurutan bergabung dengan \textit{pattern} terpilih lama untuk digunakan dalam proses \textit{pattern matching}. Hal ini untuk membuka kemungkinan \textit{pair} baru terekstraksi. Sama seperti proses sebelumnya, \textit{pair} yang lolos \textit{threshold} digabung ke dalam korpus.

Iterasi dilakukan hingga korpus pasangan kata relasi jenuh atau dapat dikatakan \textit{pair} baru yang masuk ke dalam korpus berjumlah kurang dari lima puluh. Evaluasi \textit{pattern} dan \textit{pair} dilakukan secara kolektif di akhir pengembangan. Anotator melakukan evaluasi secara manual untuk mengetahui kualitas \textit{pattern} dan \textit{pair} yang dihasilkan.

