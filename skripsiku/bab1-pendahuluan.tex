%!TEX root = skripsi.tex
%-----------------------------------------------------------------------------%
\chapter{\babSatu}
%-----------------------------------------------------------------------------%
Bab ini membahas mengenai latar belakang penelitian, perumusan masalah, tujuan dan manfaat dilakukannya penelitian, ruang lingkup penelitan, dan terakhir metodologi yang digunakan. 


%-----------------------------------------------------------------------------%
%-----------------------------------------------------------------------------%
\section{Latar Belakang}
%-----------------------------------------------------------------------------%
Relasi kata adalah salah satu informasi penting yang dibutuhkan jika ingin mengetahui hubungan antar kata. Informasi tersebut sering digunakan untuk menunjang penelitian populer lainnya yaitu \textit{Question Answering} atau sistem tanya jawab. Suatu pertanyaan dapat langsung dijawab jika diketahui hubungan antar suatu kata dalam pertanyaan dengan relasi yang ditanyakan. Sebagai contoh jika diberi pertanyaan `Apa ibu kota Indonesia?' dan dimiliki suatu \textit{resource} yang menyimpan relasi ibu-kota(Indonesia, Jakarta) maka pertanyaan tersebut langsung dapat dijawab dengan jawaban `Jakarta'.

Salah satu subbagian dalam proses pengembangan sistem tanya jawab tersebut adalah \textit{answer type detection} yang berusaha untuk mencari tahu tipe jawaban sehingga dapat menurunkan lama proses pencarian \citep{jurafsky2000speech}. Tipe jawaban dapat dibangun dalam bentuk hirarki atau yang dikenal sebagai \textit{answer type taxonomy}. Sebuah kata dalam pertanyaan mengandung informasi yang dapat mengenali tipe jawaban. Mengetahui hipernim dan hiponim kata tersebut dapat membantu mencari tahu tipe jawaban.

Berikut adalah beberapa contoh pertanyaan yang jawabannya dengan memanfaatkan informasi relasi semantik dari \textit{question headwords}-nya.

Q1. What are aquatic vertebrates?

Q2. Which type of cat that has short hair and blue eyes?

\noindent Pertanyaan pertama ingin mengetahui apa saja vertebrata yang hidup di air, sementara pertanyaan kedua ingin mengetahui jenis kucing yang berbulu pendek dan bermata biru. Pada pertanyaan pertama, \textit{question headword}-nya adalah `aquatic vertebrate'. Jika dilihat menggunakan salah satu kamus online populer, yaitu WordNet, dapat langsung diketahui jawabanya adalah `fish' (ikan) melalui relasi semantik yang dimiliki oleh kata tersebut. Untuk pertanyaan kedua dapat diketahui bahwa domain jawabanya merupakan `cat' (kucing) sehingga dapat langsung didaftar jenis kucing yang memiliki ciri-ciri yang sesuai. Hal tersebut memperlihatkan bahwa relasi semantik antar kata dibutuhkan untuk mempermudah penyelesaia masalah \textit{question answering}. 

Resource pasangan kata relasi semantik dalam Bahasa Inggris dapat diambil dari salah satu kamus digital populer yaitu WordNet \citep{miller1995wordnet}. WordNet tersebut dibangun secara manual oleh berbagai ahli linguistik. Setiap \textit{entry} pada WordNet disimpan dalam bentuk set sinonim atau biasa disebut \textit{synset} dan arti dari \textit{synset} tersebut atau biasa disebut \textit{sense}. \textit{Entry} dalam WordNet juga mengandung informasi mengenai relasi semantik antar \textit{synset}. Penelitian mengenai pembangunan WordNet Bahasa Indonesia telah dilakukan sebelumnya, sayangnya jumlah \textit{entry} dalam WordNet Bahasa Indonesia masih sangat terbatas. Selain itu, relasi semantik dalam WordNet juga belum dapat dikatakan baik. Mengetahui hal tersebut, penelitian ini berusaha membangun korpus pasangan kata relasi Behasa Indonesia.

Penelitian ini berusaha mengekstrak kata berdasarkan relasi tertentu dari suatu dokumen sehingga dihasilkan korpus pasanagan relasi kata. Fokus relasi dalam penelitian ini adalah relasi semantik hiperim-hiponim. Keduanya menyatakan relasi antara kata yang lebih umum (hipernim) dengan kata yang lebih khusus (hiponim). Berbagai penelitian sebelumnya sudah pernah mencoba mengekstrak relasi kata hipernim-hiponim dengan berbagai metode. Pada penelitian ini, metode yang digunakan adalah \textit{pattern extraction} dan \textit{matching} dengan memanfaatkan korpus Wikipedia Bahasa Indonesia. Wikipedia memuat banyak kata dari berbagai domain sehingga dapat dimanfaatkan untuk membuat \textit{pattern} yang general serta menghasilkan korpus berukuran besar.

% Berbagai penelitian yang berkaitan dengan \textit{Information Retrieval} dan \textit{Natural Language Processing} mulai muncul menggunakan Bahasa Indonesia. Salah satu yang Informasi mengenai kata dan artinya sangat dibutuhkan. Sayangnya, \textit{resource} yang dimiliki berbasis Bahasa Indonesia masih sangat terbatas. Informasi tersebut umumnya disimpan dalam kamus digital seperti WordNet. Penelitian ini dilaksanakan untuk mengembangkan \textit{resource} dalam Bahasa Indonesia. 

% WordNet adalah kamus digital yang dapat digunakan untuk menunjang penelitian di bidang \textit{Information Retrieval}dan \textit{Natural Language Processing}. WordNet yang paling sering digunakan dalam berbagai penelitian adalah WordNet Princeton berbahasa Inggris yang dibentuk secara manual oleh ahli linguistik. Setiap \textit{entry} pada WordNet disimpan dalam bentuk set sinonim atau biasa disebut \textit{synset} dan arti dari \textit{synset} tersebut atau biasa disebut \textit{sense}. Informasi lain yang disimpan dalam suatu synset adalah relasi antar synset. Relasi semantik yang tersimpan dalam WordNet adalah \textit{synonym}, \textit{antonym}, hipernim, hiponim, \textit{holonym}, \textit{meronym}, \textit{troponym}, dan \textit{entailment}.

% Beberapa penelitian telah dilakukan untuk membangun WordNet Bahasa Indonesia. WordNet Bahasa Indonesia yang telah ada adalah Indonesian WordNet (IWN) yang dibuat oleh Fakultas Ilmu Komputer Universitas Indonesia (Fasilkom UI) dan WordNet Bahasa yang dibuat oleh Nanyang Technology University (NTU). Salah satu kelemahan kedua WordNet tersebut adalah ukurannya yang masih sangat terbatas. Selain itu, informasi mengenai relasi kata juga belum dapat tersimpan secara baik. Kedua WordNet memetakan \textit{synset} Bahasa Indonesia ke WordNet Princeton dan memanfaatkan relasi yang di dalamnya. Mengetahui hal tersebut, dilakukan penelitian yang dapat mengekstrak relasi antar kata dengan hanya menggunakan korpus Bahasa Indonesia. Proses ekstraksi berjalan secara cepat dan data yang dihasilkan berjumlah besar. Korpus yang dihasilkan diharapkan dapat berguna untuk penelitian selanjutnya.

% Relasi kata adalah salah satu hal penting yang perlu diketahui jika ingin mengetahui hubungan antar kata secara semantik. Informasi yang berkaitan dengan semantik atau arti kata sulit diperoleh tanpa adanya pengetahuan sebelumnya. Kata-kata yang mirip secara leksikal belum tentu berelasi secara semantik. Sementara kata-kata yang tidak memiliki kesamaan secara leksikal bisa memiliki arti yang mirip atau berhubungan secara semantik. Korpus relasi yang dibuat diharapkan dapat membantu memperoleh informasi tersebut. Selain itu, pengetahuan mengenai relasi kata dapat dimaanfatkan dalam berbagai penelitian lain seperti \textit{question answering} \citep{ravichandran2002learning}, \textit{information extraction}, dan \textit{anaphora resolution}.

% Melihat adanya kebutuhan akan korpus relasi kata, dilakukanlah penelitian \textit{word relation extraction}. Penelitian ini berusaha mengekstrak kata berdasarkan relasi tertentu dari suatu dokumen sehingga dihasilkan korpus relasi kata. Penelitian kali ini akan fokus pada relasi kata \textit{hypernym-hyponym}. Keduanya menyatakan relasi antara kata yang lebih umum (hipernim) dengan kata yang lebih khusus (hiponim). Metode yang digunakan adalah pattern matching dengan memanfaatkan korpus Wikipedia Bahasa Indonesia. Wikipedia memuat banyak kata dari berbagai domain sehingga dapat dimanfaatkan untuk membuat pattern yang general serta menghasilkan korpus relasi berukuran besar.

%-----------------------------------------------------------------------------%
%-----------------------------------------------------------------------------%
\section{Perumusan Masalah}
%-----------------------------------------------------------------------------%
Berdasarkan latar belakang yang telah dipaparkan, pertanyaan yang menjadi rumusan penelitian adalah sebagai berikut.
\begin{enumerate}
	\item Bagaimana cara membangun korpus pasangan kata relasi hypernim-hiponim berkualitas baik secara otomatis?
	\item Seberapa baik metode \textit{pattern extraction} dan \textit{matching} digunakan untuk ekstraksi relasi semantik kata Bahasa Indonesia?
	% \item Bagaimana cara mengevaluasi \textit{pattern} dan korpus relasi kata dari hasil eksperimen?
\end{enumerate}

%-----------------------------------------------------------------------------%
%-----------------------------------------------------------------------------%
\section{Tujuan dan Manfaat Penelitian}
%-----------------------------------------------------------------------------%
Tujuan dari penelitian \textit{word relation extraction} ini adalah membangun korpus relasi kata Bahasa Indonesia berukuran besar dan berkualitas baik secara otomatis. Selain itu, ingin diketahui pula apakah metode \textit{pattern extraction} dan \textit{matching} baik digunakan untuk mengekstrak relasi kata Bahasa Indonesia. Diharapkan korpus relasi kata yang dihasilkan dapat menunjang berbagai penelitian selanjutnya.
 
Penelitian ini juga diharapkan dapat memotivasi adanya penelitian selanjutnya di bidang \textit{Language Resource Development}, terutama pembangunan WordNet Bahasa Indonesia. Penelitian mengenai ekstraksi relasi kata berikutnya dengan berbagai metode lain diharapkan terus dilaksanakan sehingga Bahasa Indonesia memiliki \textit{resource} yang semakin baik.

%-----------------------------------------------------------------------------%
%-----------------------------------------------------------------------------%
\section{Ruang Lingkup Penelitian}
%-----------------------------------------------------------------------------%
Penelitian ini hanya fokus pada pembuatan korpus pasangan kata dengan relasi hipernim dan hiponim Bahasa Indonesia. Kelas kata yang menjadi fokusi penelitan adalah kata benda (\textit{noun}). Pengembangan korpus dilakukan secara \textit{semi-supervised} dengan metode \textit{pattern matching}. \textit{Pattern} yang dibuat masih terbatas hanya merupakan \textit{pattern} leksikal. Evaluasi akan dilakukan pada \textit{pattern} dan korpus pasangan kata yang dihasilkan. Proses evaluasi korpus pasangan kata dilakukan menggunakan teknik \textit{random sampling}. Data yang digunakan untuk pembuatan \textit{pattern} maupun untuk ekstraksi korpus pasangan kata baru adalah Wikipedia Bahasa Indonesia. 
Pemilihan relasi \textit{hypernym-hyponym} dalam penelitian karena memiliki berbagai manfaat. Relasi tersebut dapat mengidentifikasi \textit{Proper Noun} baru yang belum diketahui maknanya.

%-----------------------------------------------------------------------------%
%-----------------------------------------------------------------------------%
\section{Tahapan Penelitian}
%-----------------------------------------------------------------------------%
Proses penelitian dilakukan dalam beberapa tahapan sebagai berikut.
\begin{enumerate}
	\item Studi Literatur \\
	Pada tahap ini, dilakukan pembelajaran mengenai penelitian-penelitian sebelumnya yang telah dilakukan di bidang \textit{word relation extraction} sehingga diketahui langkah yang perlu diambil selanjutnya.
	\item Perumusan Masalah \\
	Perumusan masalah dilakukan untuk mendefinisikan masalah yang ingin diselesaikan, tujuan penelitian, dan hasil yang diharapkan sehingga proses penelitian dapat berjalan dengan baik.
	\item Perancangan Penelitian \\
	Setelah diketahui hasil yang ingin dicapai, dirancang tahap-tahap eksperimen secara terstruktur. Hal-hal yang diperhatikan mulai dari pengumpulan korpus awal (\textit{seed}), \textit{pre-processing} dokumen, perancangan implementasi \textit{pattern extraction} \textit{matching}, hingga proses evaluasi.
	\item Implementasi \\
	Implementasi dilaksanakan sesuai dengan rancangan penelitian untuk menjawab rumusan masalah. Segala hasil yang ditemukan digunakan untuk terus memperbaiki metode dan teknik penelitan sehingga didapatkan hasil yang semakin baik.
	\item Analisis dan Kesimpulan \\
	Tahap terakhir dari penelitian ini adalah menganalisis korpus pasangan kata relasi yang dihasilkan. Pertanyaan dari perumusan masalah dijawab, kemudian ditarik kesimpulan.
	
\end{enumerate}


%-----------------------------------------------------------------------------%
%\section{Sistematika Penulisan}
%-----------------------------------------------------------------------------%
% Sistematika penulisan laporan adalah sebagai berikut:
% \begin{itemize}
% 	\item Bab 1 \babSatu \\
% 	\item Bab 2 \babDua \\
% 	\item Bab 3 \babTiga \\
% 	\item Bab 4 \babEmpat \\
% 	\item Bab 5 \babLima \\
% 	\item Bab 6 \babEnam \\
% 	\item Bab 7 \babTujuh \\
% \end{itemize}

